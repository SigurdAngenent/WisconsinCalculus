%%%%%%%%%%%%  Self made chapter headings %%%%%%%%%%
%  Section numbering does NOT get reset 
%  in every new chapter.
%%%%%%%%%%%%%%%%%%%%%%%%%%%%%%%%%%%%%%%%%%%%%%%%%%%
%%\newcommand{\chapter}[1]{\newpage
%%~
%%\vspace{24pt}
%%\begin{center}
%%    \large\sffamily\bfseries#1
%%\end{center}
%%\vspace{24pt}
%%\addcontentsline{toc}{part}{#1}
%%\setcounter{secno}{0}
%%}
%\newcommand{\chapter}[1]{\bigskip
%
%  \begin{center}
%    \LARGE\sffamily\bfseries#1
%  \end{center}\bigskip
%
%}
%%%%%%%%%%%%%%%%%%%%%%%%%%%%%%%%%%%
%       ENVIRONMENTS
%%%%%%%%%%%%%%%%%%%%%%%%%%%%%%%%%%%
\newenvironment{definition}[1][Definition]
{\subsection{#1}\itshape}
{\upshape}
\newenvironment{theorem}[1][Theorem]
{\subsection{#1}\itshape}
{\upshape}
\newenvironment{lemma}[1][Lemma]
{\subsection{#1}\itshape}
{\upshape}
%%%%%%%%%%%%%%%%%%%%%%%%%%%%%%%%%%%
%      MACROS for PROBLEMS
%%%%%%%%%%%%%%%%%%%%%%%%%%%%%%%%%%%
\newcommand{\problemstyle}{\sffamily\small}
\newcommand{\noproblemstyle}{\rmfamily\normalsize}

\newcounter{PROB}
\newcommand{\prob}{\refstepcounter{PROB}%
\noindent\hspace{1em}\makebox[0pt][r]{\textbf{\thePROB.~}}
\ignorespaces}

\newcommand{\problem}{\par\smallskip

\noindent\prob}

\newcounter{SUBPROB}[PROB]
\newcommand{\subprob}
 {\refstepcounter{SUBPROB}\noindent(\textbf{\slshape\roman{SUBPROB}})~~}
%
\input gans.tex


%%%%%%%%%%%%%%%%%%%%%%%%%%%%%%%%%%%%%%%%%%%%%%%%%%%%%%%%%%%
%% GENERAL MACROS
\newcommand{\DS}{\displaystyle}
%% TAYLOR MACROS
\newcommand{\limntoi}{\lim_{n\to\infty}}
\newcommand{\Ti}{T_{\infty}}
%% COMPLEX NUMBER MACROS
\renewcommand{\Re}{\mathfrak{Re}}
\renewcommand{\Im}{\mathfrak{Im}}
%% VECTOR MACROS, used in 222 and 234.
\newcommand{\vvv}[1]{{\vec{\boldsymbol #1}}}  % Use for lower case
\newcommand{\VVV}[1]{{\vec{\mathbf{#1}}}}     % Use for uppercase
% if \boldsymbol gives problems use this :
% \newcommand{\vvv}[1]{\smash{{{\vec{\text{\bfseries\itshape{#1}}}}\,}}}
%%%% JOEL'S vector and matrix macros
\newcommand{\mat}{\begin{pmatrix}}
\newcommand{\rix}{\end{pmatrix}}
\newcommand{\tmat}{\left(\begin{smallmatrix}}
\newcommand{\trix}{\end{smallmatrix}\right)}
\newcommand{\vek}{\mat}
\newcommand{\tor}{\rix}
\newcommand{\tvek}{\tmat}
\newcommand{\ttor}{\trix}
\newcommand{\tpv}[2]{\overrightarrow{#1 #2}} % Two Point Vector, e.g. $\tpv AB$
                                             % gives the vector from A to B.
\newcommand{\va}{{\vvv a}}
\newcommand{\vb}{{\vvv b}}
\newcommand{\vc}{{\vvv c}}
\newcommand{\vd}{{\vvv d}}
\newcommand{\ve}{{\vvv e}}
\newcommand{\vf}{{\vvv f}}
\newcommand{\vg}{{\vvv g}}
\newcommand{\vp}{{\vvv p}}
\newcommand{\vi}{{\vvv \imath}}
\newcommand{\vj}{{\vvv \jmath}}
\newcommand{\vk}{{\vvv k}}
\newcommand{\vm}{{\vvv m}}
\newcommand{\vn}{{\vvv n}}
\newcommand{\vq}{{\vvv q}}
\newcommand{\vr}{{\vvv r}}
\newcommand{\vs}{{\vvv s}}
\newcommand{\vt}{{\vvv t}}
\newcommand{\vu}{{\vvv u}}
\newcommand{\vv}{{\vvv v}}
\newcommand{\vw}{{\vvv w}}
\newcommand{\vx}{{\vvv x}}
\newcommand{\vy}{{\vvv y}}
\newcommand{\vz}{{\vvv z}}
\newcommand{\vB}{{\VVV B}}
\newcommand{\vE}{{\VVV E}}
\newcommand{\vF}{{\VVV F}}
\newcommand{\vL}{{\VVV L}}
\newcommand{\vN}{{\VVV N}}
\newcommand{\vT}{{\VVV T}}
\newcommand{\dpp}{\pmb{\cdot}}        % dot product
\newcommand{\cp}{\pmb{\times}}        % cross product
\newcommand{\parppd}[8]{%               parallelepiped
  {#1#2#3#4 \atop #5#6#7#8}
}
\newcommand{\nm}[1]{\left\|\smash{#1}\right\|}
\newcommand{\Nm}[1]{\left\|{#1}\right\|}


\newcommand{\odiff}[2]{\frac{\dd #1}{\dd #2}}
\newcommand{\pdd}[2]{\frac{\partial #1}{\partial #2}}
\newcommand{\isdef}{\stackrel{\mathrm{def}}{=}}
\newcommand{\pll}{\|}

\renewcommand{\emph}[1]{{\bfseries\itshape #1}}
\newcommand{\N}{\mathbb{N}}
\newcommand{\Z}{\mathbb{Z}}
\newcommand{\R}{\mathbb{R}}
\newcommand{\C}{\mathbb{C}}
\newcommand{\cC}{\mathcal{C}} 
\newcommand{\cL}{\mathcal{L}} % used for differential operators in 222
\newcommand{\cP}{\mathcal{P}} % used for planes in Euclidean space
\newcommand{\cD}{\mathcal{D}} % used for open subsets, domains, regions, etc.
\newcommand{\cO}{\mathcal{O}} 
\newcommand{\cR}{\mathcal{R}} 
\newcommand{\pd}{\partial}    % partial differentials
\newcommand{\nab}{\vvv\nabla}
\renewcommand{\div}{\mathop{\rm div}}
\newcommand{\grad}{\mathop{\bf grad}}
\newcommand{\curl}{\mathop{\bf curl}}
\newcommand{\rot}{\mathop{\bf rot}}
\newcommand{\lint}{\int\limits}
\newcommand{\liint}{\iint\limits}
\newcommand{\liiint}{\iiint\limits}

\newcommand{\hC}{{\hat C}}
\newcommand{\toi}{\to\infty}

\newcommand{\ssum}[3]{\sum_{#1=#2}^{#3}}
\newcommand{\tsum}{{\textstyle\sum}}
\renewcommand{\r}{\right}
\renewcommand{\l}{\left}
\newcommand{\cis}[1]{\cos#1+i\sin#1}
\newcommand{\artanh}{\mathop{\mathrm{artanh}}}
\newcommand{\dV}{\mathrm{d}V}

%%% Local Variables: 
%%% mode: latex
%%% TeX-master: "free234"
%%% End: 
