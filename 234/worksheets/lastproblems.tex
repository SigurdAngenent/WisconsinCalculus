\documentclass{amsart}
%% PACKAGES {{{1
\usepackage{amsmath}
\usepackage{amssymb}
\usepackage{graphicx}

\usepackage{eucal}


\newcommand\version{0.9}
\newcommand\semester{Fall 2009}
%%%%%%%%%%%%  Self made chapter headings %%%%%%%%%%
%  Section numbering does NOT get reset 
%  in every new chapter.
%%%%%%%%%%%%%%%%%%%%%%%%%%%%%%%%%%%%%%%%%%%%%%%%%%%
%%\newcommand{\chapter}[1]{\newpage
%%~
%%\vspace{24pt}
%%\begin{center}
%%    \large\sffamily\bfseries#1
%%\end{center}
%%\vspace{24pt}
%%\addcontentsline{toc}{part}{#1}
%%\setcounter{secno}{0}
%%}
%\newcommand{\chapter}[1]{\bigskip
%
%  \begin{center}
%    \LARGE\sffamily\bfseries#1
%  \end{center}\bigskip
%
%}
%%%%%%%%%%%%%%%%%%%%%%%%%%%%%%%%%%%
%       ENVIRONMENTS
%%%%%%%%%%%%%%%%%%%%%%%%%%%%%%%%%%%
\newenvironment{definition}[1][Definition]
{\subsection{#1}\itshape}
{\upshape}
\newenvironment{theorem}[1][Theorem]
{\subsection{#1}\itshape}
{\upshape}
\newenvironment{lemma}[1][Lemma]
{\subsection{#1}\itshape}
{\upshape}
%%%%%%%%%%%%%%%%%%%%%%%%%%%%%%%%%%%
%      MACROS for PROBLEMS
%%%%%%%%%%%%%%%%%%%%%%%%%%%%%%%%%%%
\newcommand{\problemstyle}{\sffamily\small}
\newcommand{\noproblemstyle}{\rmfamily\normalsize}

\newcounter{PROB}
\newcommand{\prob}{\refstepcounter{PROB}%
\noindent\hspace{1em}\makebox[0pt][r]{\textbf{\thePROB.~}}
\ignorespaces}

\newcommand{\problem}{\par\smallskip

\noindent\prob}

\newcounter{SUBPROB}[PROB]
\newcommand{\subprob}
 {\refstepcounter{SUBPROB}\noindent(\textbf{\slshape\roman{SUBPROB}})~~}
%
\input gans.tex


%%%%%%%%%%%%%%%%%%%%%%%%%%%%%%%%%%%%%%%%%%%%%%%%%%%%%%%%%%%
%% GENERAL MACROS
\newcommand{\DS}{\displaystyle}
%% TAYLOR MACROS
\newcommand{\limntoi}{\lim_{n\to\infty}}
\newcommand{\Ti}{T_{\infty}}
%% COMPLEX NUMBER MACROS
\renewcommand{\Re}{\mathfrak{Re}}
\renewcommand{\Im}{\mathfrak{Im}}
%% VECTOR MACROS, used in 222 and 234.
\newcommand{\vvv}[1]{{\vec{\boldsymbol #1}}}  % Use for lower case
\newcommand{\VVV}[1]{{\vec{\mathbf{#1}}}}     % Use for uppercase
% if \boldsymbol gives problems use this :
% \newcommand{\vvv}[1]{\smash{{{\vec{\text{\bfseries\itshape{#1}}}}\,}}}
%%%% JOEL'S vector and matrix macros
\newcommand{\mat}{\begin{pmatrix}}
\newcommand{\rix}{\end{pmatrix}}
\newcommand{\tmat}{\left(\begin{smallmatrix}}
\newcommand{\trix}{\end{smallmatrix}\right)}
\newcommand{\vek}{\mat}
\newcommand{\tor}{\rix}
\newcommand{\tvek}{\tmat}
\newcommand{\ttor}{\trix}
\newcommand{\tpv}[2]{\overrightarrow{#1 #2}} % Two Point Vector, e.g. $\tpv AB$
                                             % gives the vector from A to B.
\newcommand{\va}{{\vvv a}}
\newcommand{\vb}{{\vvv b}}
\newcommand{\vc}{{\vvv c}}
\newcommand{\vd}{{\vvv d}}
\newcommand{\ve}{{\vvv e}}
\newcommand{\vf}{{\vvv f}}
\newcommand{\vg}{{\vvv g}}
\newcommand{\vp}{{\vvv p}}
\newcommand{\vi}{{\vvv \imath}}
\newcommand{\vj}{{\vvv \jmath}}
\newcommand{\vk}{{\vvv k}}
\newcommand{\vm}{{\vvv m}}
\newcommand{\vn}{{\vvv n}}
\newcommand{\vq}{{\vvv q}}
\newcommand{\vr}{{\vvv r}}
\newcommand{\vs}{{\vvv s}}
\newcommand{\vt}{{\vvv t}}
\newcommand{\vu}{{\vvv u}}
\newcommand{\vv}{{\vvv v}}
\newcommand{\vw}{{\vvv w}}
\newcommand{\vx}{{\vvv x}}
\newcommand{\vy}{{\vvv y}}
\newcommand{\vz}{{\vvv z}}
\newcommand{\vB}{{\VVV B}}
\newcommand{\vE}{{\VVV E}}
\newcommand{\vF}{{\VVV F}}
\newcommand{\vL}{{\VVV L}}
\newcommand{\vN}{{\VVV N}}
\newcommand{\vT}{{\VVV T}}
\newcommand{\dpp}{\pmb{\cdot}}        % dot product
\newcommand{\cp}{\pmb{\times}}        % cross product
\newcommand{\parppd}[8]{%               parallelepiped
  {#1#2#3#4 \atop #5#6#7#8}
}
\newcommand{\nm}[1]{\left\|\smash{#1}\right\|}
\newcommand{\Nm}[1]{\left\|{#1}\right\|}


\newcommand{\odiff}[2]{\frac{\dd #1}{\dd #2}}
\newcommand{\pdd}[2]{\frac{\partial #1}{\partial #2}}
\newcommand{\isdef}{\stackrel{\mathrm{def}}{=}}
\newcommand{\pll}{\|}

\renewcommand{\emph}[1]{{\bfseries\itshape #1}}
\newcommand{\N}{\mathbb{N}}
\newcommand{\Z}{\mathbb{Z}}
\newcommand{\R}{\mathbb{R}}
\newcommand{\C}{\mathbb{C}}
\newcommand{\cC}{\mathcal{C}} 
\newcommand{\cL}{\mathcal{L}} % used for differential operators in 222
\newcommand{\cP}{\mathcal{P}} % used for planes in Euclidean space
\newcommand{\cD}{\mathcal{D}} % used for open subsets, domains, regions, etc.
\newcommand{\cO}{\mathcal{O}} 
\newcommand{\cR}{\mathcal{R}} 
\newcommand{\pd}{\partial}    % partial differentials
\newcommand{\nab}{\vvv\nabla}
\renewcommand{\div}{\mathop{\rm div}}
\newcommand{\grad}{\mathop{\bf grad}}
\newcommand{\curl}{\mathop{\bf curl}}
\newcommand{\rot}{\mathop{\bf rot}}
\newcommand{\lint}{\int\limits}
\newcommand{\liint}{\iint\limits}
\newcommand{\liiint}{\iiint\limits}

\newcommand{\hC}{{\hat C}}
\newcommand{\toi}{\to\infty}

\newcommand{\ssum}[3]{\sum_{#1=#2}^{#3}}
\newcommand{\tsum}{{\textstyle\sum}}
\renewcommand{\r}{\right}
\renewcommand{\l}{\left}
\newcommand{\cis}[1]{\cos#1+i\sin#1}
\newcommand{\artanh}{\mathop{\mathrm{artanh}}}
\newcommand{\dV}{\mathrm{d}V}

%%% Local Variables: 
%%% mode: latex
%%% TeX-master: "free234"
%%% End: 

%%}}}

\begin{document}
\title{Vector calculus problems}
\maketitle
\problem \label{prb:05derivs-of-a-times-m-dot-x} % {{{1
Let $\va$ and $\vm$ be two constant vectors, with components
\[
\va = \tvek a_1\\a_2\\a_3\ttor, \text{ and }
\vm = \tvek m_1\\m_2\\m_3\ttor.
\]
Let $\vv(x, y, z)$ be the vector field $\vv = (\vm\dpp\vx)\va$.

\subprob Write $\vv$ in terms of its components:
\[
\vv = \vek \cdots?\cdots \\ \cdots?\cdots \\ \cdots?\cdots \tor.
\]
\subprob Compute $\nab\dpp \vv$. % {{{3

\subprob Compute $\nab \cp \vv$. % {{{3

\subprob If $\vv$ is the gradient of some function $f$, % {{{3
what can you say about the vectors $\va$ and $\vm$?

\subprob If $\vv$ is the curl of some vector field $\vw$, % {{{3
what can you say about the vectors $\va$ and $\vm$?

\problem \label{prb:05derivs-of-a-exp-of-mx} % {{{1
Let $\va$ and $\vm$ be as in the previous problem.
Consider the vector field
\[
\vv(x, y, z) = e^{\vm\dpp\vx}\va = e^{m_1x+m_2y+m_3z}\tvek a_1\\a_2\\a_3\ttor.
\]
\subprob Show by computing the derivatives %{{{3
that $\nab \bigl(e^{\vm\dpp\vx}\bigr) = e^{\vm\dpp\vx} \vm$.
\answer
$\bigl(e^{\vm\dpp\vx}\bigr)_{x_1} = m_1e^{\vm\dpp\vx}$, and the same for the $x_2$
and $x_3$ derivatives.  Therefore
\[
\nab\bigl(e^{\vm\dpp\vx}\bigr) =
\vek m_1e^{\vm\dpp\vx}\\m_2e^{\vm\dpp\vx} \\ m_3e^{\vm\dpp\vx} \tor
=e^{\vm\dpp\vx} \tvek m_1\\m_2\\m_3\ttor.
\]
\endanswer
\subprob Compute $\nab\dpp \vv$.  (Find the shortest way to write %{{{3
the answer.)
\answer
After simplifying you get $\nab\dpp\vv = \vm\dpp\va e^{\vm\dpp\vx}$.
\endanswer
\subprob Compute $\nab\cp\vv$.\label{prb:05cross-prod-of-aexpmx} %{{{3
Again, simplify your answer.
\answer
$\nab\cp\vv = \vm\cp\va e^{\vm\dpp\vx}$.
\endanswer
\subprob Which condition must the vectors $\va$ and $\vm$ satisfy if  %{{{3
$\vv$ is to be ``divergence free,'' i.e.\ if $\div \vv = 0$?
\answer
$\va$ and $\vm$ must be perpendicular.
\endanswer
\subprob Suppose that $\vv = \nab \phi$ for some function. %{{{3
What do you know about $\va$ and $\vm$?
\answer
If $\vv$ is the gradient of some function, then its curl must vanish.
Therefore $\va\cp\vm=\vvv0$ in view of part \ref{prb:05cross-prod-of-aexpmx}
of this problem.  The conclusion is that $\va$ and $\vm$ must
be parallel.
\endanswer
\problem If $\vv=\tvek P\\ Q \\ R\ttor$ is a vector field and $f$ is %{{{1
a function, then what is $\vv\dpp\nab f$?
\answer
$\vv\dpp\nab f = P f_x + Q f_y + R f_z$.
\endanswer
\problem \emph{Product rules. }\label{prb:05product-rule} %{{{1
Let $f$ be a function of three variables, and let $\vv$ be a three
dimensional vector field.

\subprob $\nab\dpp(f\vv) = (\nab f)\dpp\vv + f \nab\dpp\vv$ %{{{3
\answer
By definition,
\begin{align*}
  \nab\dpp(f\vv) = \nab \dpp \vek fP\\fQ \\fR\tor
  &= \pdd{fP}x + \pdd{fQ}y + \pdd{fR}z  \\
  &= f_x P + fP_x + f_yQ + fQ_y + f_z R + fR_z\\
  &= f_x P +f_y Q + f_z R + f\bigl(P_x+Q_y+R_z\bigr) \\
  &= \vek f_x\\ f_y\\ f_z\tor \dpp \vek P\\Q\\R\tor + f \nab\dpp\vv \\
  &= \nab f\dpp \vv + f\nab \dpp\vv,
\end{align*}
as claimed.
\endanswer
\subprob Guess a product rule for $\nab\cp(f\vv)$ and prove it. %{{{3
\answer
$\nab\cp(f\vv) = (\nab f)\cp\vv + f\nab\cp\vv$ is the rule.
The derivation goes along the same lines as in the previous product rule.
\endanswer
\problem \label{prb:gradrho-and-divx} Check the following %{{{1
formulas
\[
\nab \rho = \frac{\vx} {\rho}, \text{ and }
\nab\dpp \vx = 3.
\]
by direct computation.
Here $\rho = \sqrt{x^2 + y^2 + z^2}$ is the radius from spherical
coordinates.

\problem Use the product rule from Problem~\ref{prb:05product-rule} and %{{{1
the formulas from problem \ref{prb:gradrho-and-divx} to compute the
following quantities

\subprob $\nab\dpp(\rho^2\vx)$ \qquad
\subprob $\vx\dpp\nab\rho$ \qquad
\subprob $\vx\dpp\nab\bigl(\|\vx\|\bigr)$ \qquad
\subprob $\DS \div \frac{\vx} {\|\vx\|^3}$.  



\problem \label{prb:x-no-curl} %{{{1
In this problem, as in all the problems in this section,
$\rho = \sqrt{x^2+y^2+z^2} = \|\vx\|$
is the radius in spherical coordinates.

\subprob Show that $\vx = \frac12 \nab(\rho^2)$. %{{{3

\subprob Compute $\nab\cp\vx$ without doing any derivatives. %{{{3
\answer
Since $\vx$ is the gradient of some function its curl must vanish.
\endanswer
\subprob Compute $\nab\cp(\rho\vx)$ using the product rule from %{{{3
problem \ref{prb:05product-rule}.
\answer
$\nab\cp(\rho\vx) = (\nab \rho)\cp \vx + \rho \nab\cp\vx = \vvv0$
\endanswer
\problem Compute $\nab\cp\vv$ for the vector field %{{{1
$\vv(x, y, z) = \vk\cp\vx$.
\answer
$\vv(x, y, z) = \tvek -y \\x \\ 0\ttor$
so $\nab\cp\vv = \tvek 0\\0\\2\ttor = 2\vk$.
\endanswer
\problem\label{prb:05div-rho-n-x} Consider the vector field
\[
\vv(x, y, z) = \rho^n \vx,
\]
where $n$ is a constant.  (Both Newton's law of gravitation
and Coulomb's law have this vector field with $n=-3$.)

\subprob Write $\vv(x, y, z)$ in the form $\tvek \cdots\\\cdots\\\cdots\ttor$, %{{{3
using only Cartesian coordinates $x, y, z$.
\answer
$\vv(x, y, z) =
\vek
   x(x^2+y^2+z^2)^{n/2} \\ y(x^2+y^2+z^2)^{n/2} \\ z(x^2+y^2+z^2)^{n/2}
\tor$.
\endanswer


\subprob Compute $\nab\dpp\vv$.  (Use one of the product %{{{3
rules from Problem \ref{prb:05product-rule}; you already
computed the derivatives of $\rho$ in problem \ref{prb:gradrho-and-divx}.)
\answer
Using the product rule, you get
\[
\nab(\rho^{n}\vx)
= (\nab \rho^{n})\dpp\vx + \rho^{n}\nab\dpp\vx
= n\rho^{n-1}(\nab\rho)\dpp\vx + \rho^{n}\nab\dpp\vx.
\]
Now recall (or compute again):
\[
\nab \rho = \frac{\vx} {\rho}, \text{ and }
\nab\dpp\vx = 3.
\]
This leads to
\[
\nab(\rho^{n}\vx)
= - n \rho^{n-1}\frac{\vx} {\rho}\dpp\vx + 3 \rho^{n}
= - n \rho^{n-2}\|\vx\|^2 + 3 \rho^{n}
= (n+3) \rho^{-n})
\]
\endanswer
\subprob For which value(s) of $n$ does one have $\div \vv = 0$? %{{{3
\answer
$n=-3$.
\endanswer
\problem\label{prb:05grad-Frho} A function of three variables % {{{1
is called \emph{radially symmetric}
if it only depends on the radius $\rho = \sqrt{x^2+y^2+z^2}$, i.e.\
if it can be written as $F(\rho)$ for some function $F$ of one variable.
E.g.\ $f(x, y, z) = \rho^{-2}$, or $g(x, y, z) = e^{-\rho}$ are radially
symmetric functions. 


\subprob Find the gradient of a radially symmetric function $F(\rho)$. % {{{3

\answer
There are a long and a short answer. 
The long(er) computation goes likes this:
\[
\nab F(\rho)
= \vek F(\rho)_x\\ F(\rho)_y \\ F(\rho)_z\tor
= \vek F'(\rho)\rho_x\\ F'(\rho)\rho_y \\ F'(\rho)\rho_z\tor
= F'(\rho)\vek \rho_x\\\rho_y\\\rho_z\tor.
\]
Now recall problem \ref{prb:gradrho-and-divx}, and you find
\[
\nab F(\rho)
=F'(\rho)\vek x/\rho\\y/\rho\\z/\rho\tor
=\frac{1} {\rho}F'(\rho) \vx.
\]
The short computation is essentially the same, but you never
write the components of the vectors:
\[
\nab F(\rho) = F'(\rho) \nab \rho = \frac{1} {\rho}F'(\rho)\vx.
\]
\endanswer
\subprob Let $\vv = \rho^n\vx$, as in problem \ref{prb:05div-rho-n-x}. %{{{3
Does there exist a function $f(x, y, z)$ such that $\vv = \nab f$?
(Hint: try a radially symmetric function, and use problem
\ref{prb:05grad-Frho}.)
\answer
If $f(x, y, z)= F(\rho)$, then by the previous problem
we have $\nab f = \rho^{-1}F'(\rho) \vx$.  We want this to be equal to
$\rho^{-n}\vx$, so $F(\rho)$ must satisfy
\[
\rho^{-1}F'(rho) = \rho^{n} \implies
F'(\rho) = \rho^{1+n} \implies
F(\rho) = \frac{\rho^{2+n}} {2+n} +C
\]
for some constant $C$.  We are only asked to find on function $f$,
so we find that the given vector field is indeed the gradient of a radially
symmetric function:
\[
\vv = \rho^{n}\vx = \nab \bigl(\frac{\rho^{2+n}} {2+n}\bigr).
\]
The exceptional case is when $n=-2$, in which case you get
$F(\rho) = \ln \rho$.
\endanswer

\immediate\closeout\ans

\begin{trivlist}
  \input answersANDhints
\end{trivlist}

\end{document}
