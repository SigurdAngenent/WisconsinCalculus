\relax
\tolerance10000

\item[{\bf(2i)}]

$\bigl(e^{\vm\dpp\vx}\bigr)_{x_1} = m_1e^{\vm\dpp\vx}$, and the same for the $x_2$
and $x_3$ derivatives.  Therefore
\[
\nab\bigl(e^{\vm\dpp\vx}\bigr) =
\vek m_1e^{\vm\dpp\vx}\\m_2e^{\vm\dpp\vx} \\ m_3e^{\vm\dpp\vx} \tor
=e^{\vm\dpp\vx} \tvek m_1\\m_2\\m_3\ttor.
\]
\bigskip

\item[{\bf(2ii)}]

After simplifying you get $\nab\dpp\vv = \vm\dpp\va e^{\vm\dpp\vx}$.
\bigskip

\item[{\bf(2iii)}]

$\nab\cp\vv = \vm\cp\va e^{\vm\dpp\vx}$.
\bigskip

\item[{\bf(2iv)}]

$\va$ and $\vm$ must be perpendicular.
\bigskip

\item[{\bf(2v)}]

If $\vv$ is the gradient of some function, then its curl must vanish.
Therefore $\va\cp\vm=\vvv0$ in view of part \ref{prb:05cross-prod-of-aexpmx}
of this problem.  The conclusion is that $\va$ and $\vm$ must
be parallel.
\bigskip

\item[{\bf(3)}]

$\vv\dpp\nab f = P f_x + Q f_y + R f_z$.
\bigskip

\item[{\bf(4i)}]

By definition,
\begin{align*}
  \nab\dpp(f\vv) = \nab \dpp \vek fP\\fQ \\fR\tor
  &= \pdd{fP}x + \pdd{fQ}y + \pdd{fR}z  \\
  &= f_x P + fP_x + f_yQ + fQ_y + f_z R + fR_z\\
  &= f_x P +f_y Q + f_z R + f\bigl(P_x+Q_y+R_z\bigr) \\
  &= \vek f_x\\ f_y\\ f_z\tor \dpp \vek P\\Q\\R\tor + f \nab\dpp\vv \\
  &= \nab f\dpp \vv + f\nab \dpp\vv,
\end{align*}
as claimed.
\bigskip

\item[{\bf(4ii)}]

$\nab\cp(f\vv) = (\nab f)\cp\vv + f\nab\cp\vv$ is the rule.
The derivation goes along the same lines as in the previous product rule.
\bigskip

\item[{\bf(7ii)}]

Since $\vx$ is the gradient of some function its curl must vanish.
\bigskip

\item[{\bf(7iii)}]

$\nab\cp(\rho\vx) = (\nab \rho)\cp \vx + \rho \nab\cp\vx = \vvv0$
\bigskip

\item[{\bf(8)}]

$\vv(x, y, z) = \tvek -y \\x \\ 0\ttor$
so $\nab\cp\vv = \tvek 0\\0\\2\ttor = 2\vk$.
\bigskip

\item[{\bf(9i)}]

$\vv(x, y, z) =
\vek
   x(x^2+y^2+z^2)^{n/2} \\ y(x^2+y^2+z^2)^{n/2} \\ z(x^2+y^2+z^2)^{n/2}
\tor$.
\bigskip

\item[{\bf(9ii)}]

Using the product rule, you get
\[
\nab(\rho^{n}\vx)
= (\nab \rho^{n})\dpp\vx + \rho^{n}\nab\dpp\vx
= n\rho^{n-1}(\nab\rho)\dpp\vx + \rho^{n}\nab\dpp\vx.
\]
Now recall (or compute again):
\[
\nab \rho = \frac{\vx} {\rho}, \text{ and }
\nab\dpp\vx = 3.
\]
This leads to
\[
\nab(\rho^{n}\vx)
= - n \rho^{n-1}\frac{\vx} {\rho}\dpp\vx + 3 \rho^{n}
= - n \rho^{n-2}\|\vx\|^2 + 3 \rho^{n}
= (n+3) \rho^{-n})
\]
\bigskip

\item[{\bf(9iii)}]

$n=-3$.
\bigskip

\item[{\bf(10i)}]

There are a long and a short answer.
The long(er) computation goes likes this:
\[
\nab F(\rho)
= \vek F(\rho)_x\\ F(\rho)_y \\ F(\rho)_z\tor
= \vek F'(\rho)\rho_x\\ F'(\rho)\rho_y \\ F'(\rho)\rho_z\tor
= F'(\rho)\vek \rho_x\\\rho_y\\\rho_z\tor.
\]
Now recall problem \ref{prb:gradrho-and-divx}, and you find
\[
\nab F(\rho)
=F'(\rho)\vek x/\rho\\y/\rho\\z/\rho\tor
=\frac{1} {\rho}F'(\rho) \vx.
\]
The short computation is essentially the same, but you never
write the components of the vectors:
\[
\nab F(\rho) = F'(\rho) \nab \rho = \frac{1} {\rho}F'(\rho)\vx.
\]
\bigskip

\item[{\bf(10ii)}]

If $f(x, y, z)= F(\rho)$, then by the previous problem
we have $\nab f = \rho^{-1}F'(\rho) \vx$.  We want this to be equal to
$\rho^{-n}\vx$, so $F(\rho)$ must satisfy
\[
\rho^{-1}F'(rho) = \rho^{n} \implies
F'(\rho) = \rho^{1+n} \implies
F(\rho) = \frac{\rho^{2+n}} {2+n} +C
\]
for some constant $C$.  We are only asked to find on function $f$,
so we find that the given vector field is indeed the gradient of a radially
symmetric function:
\[
\vv = \rho^{n}\vx = \nab \bigl(\frac{\rho^{2+n}} {2+n}\bigr).
\]
The exceptional case is when $n=-2$, in which case you get
$F(\rho) = \ln \rho$.
\bigskip
