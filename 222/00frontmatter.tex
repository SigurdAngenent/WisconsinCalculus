%!TEX root =  free222.tex
%%
%%%%%%%%%%%%%%%%%%%%%%%%%%%%%%%%%%%%%%%%%%%%%%%%%%
%%%             THE TITLE PAGE
\null
\includegraphics[width=0.2\textwidth]{UW_logo.png}
\vfill


\begin{flushright}
  \Huge\bfseries\sffamily%
  MATH 222 \\
  \rmfamily
  \mdseries\itshape Second Semester \\
  Calculus\\[1in]
  \large \semester
\end{flushright}

\vfill
\begin{flushright}
\textsf{Typeset:\today}
\end{flushright}

\newpage
%%%%%%%%%%%%%%%%%%%%%%%%%%%%%%%%%%%%%%%%%%%%%%%%%%
\begin{center}
  \bfseries\sffamily Math 222 -- 2nd Semester Calculus \\
  Lecture notes version \version\ (\semester)
\end{center}

\noindent This is a self contained set of lecture notes for Math 222. The
notes were written by Sigurd Angenent, as part ot the MIU calculus project.
Some problems were contributed by A.Miller.


The \LaTeX\ files, as well as the \textsc{Inkscape} and \textsc{Octave} files
which were used to produce these notes are available at the following web
site
\begin{center}
\url{http://www.math.wisc.edu/~angenent/MIU-calculus}
\end{center}
They are meant to be freely available for non-commercial use, in the sense
that ``free software'' is free. More precisely:

\bigskip

\begin{center}
  \framebox{
  \begin{minipage}[b]{4.5in}\raggedright\footnotesize
    Copyright (c) 2012 Sigurd B. Angenent.  Permission is granted to
    copy, distribute and/or modify this document under the terms of
    the GNU Free Documentation License, Version 1.2 or any later
    version published by the Free Software Foundation; with no
    Invariant Sections, no Front-Cover Texts, and no Back-Cover Texts.
    A copy of the license is included in the section entitled "GNU
    Free Documentation License".
  \end{minipage}
  }
\end{center}


\sffamily\footnotesize
%\begin{multicols}{2}
    \tableofcontents
%\end{multicols}
\rmfamily\normalsize

%  section{The table of integrals}

\begin{table}[bt]
  \centering\sffamily
  \color{darkbluegreen}%
  \rule[6pt]{\textwidth}{2pt}
  \begin{tabular}{r@{\,$=$\,}lr@{$\,=\,$}lp{36pt}}
      $\displaystyle f(x)$&
    $\displaystyle \frac{dF(x)} {dx} $ &
    $\displaystyle  \int f(x)\, d x  $&
    $\displaystyle  F(x)+C $ &
    \\[3ex]
    \midrule
    $\displaystyle (n+1)x^n $&
    $\displaystyle  \frac{dx^{n+1}} {dx}$ &
    $\displaystyle \int x^{n}\, d x  $&
    $\displaystyle   \frac{x^{n+1}}{n+1}+C$ &
    $n\neq -1$\\[3ex]
    $\displaystyle \frac{1}{x} $&
    $\displaystyle  \frac{d \ln |x|}{dx} $&
    $\displaystyle \int \frac{1}{x}\, d x $&
    $\displaystyle  \ln |x| +C $&
    absolute values!!\\[3ex]
    $\displaystyle e^x $&
    $\displaystyle  \frac{de^x} {dx}$ &
    $\displaystyle \int e^x\, dx $&
    $\displaystyle  e^x +C$& \\[3ex]
    $\displaystyle -\sin x $&
    $\displaystyle  \frac{d \cos x}{dx}$&
    $\displaystyle \int \sin x\, d x  $&
    $\displaystyle   -\cos x+C$\\[3ex]
    $\displaystyle  \cos x  $&
    $\displaystyle  \frac{d  \sin x}{dx}$&
    $\displaystyle \int \cos x\, d x  $&
    $\displaystyle   \sin x+C$\\[3ex]
    $\displaystyle  \tan x  $&
    $\displaystyle  -\frac{d \ln |\cos x|}{dx}$ &
    $\displaystyle \int \tan x $&
    $\displaystyle  -\ln |\cos x|+C$ &
    absolute values!!\\[3ex]
    $\displaystyle  \frac{1}{1+x^{2}} $&
    $\displaystyle   \frac{d \arctan x}{dx}$&
    $\displaystyle \int \frac{1}{1+x^{2}}\, d x  $&
    $\displaystyle   \arctan x+C$\\[3ex]
    $\displaystyle  \frac{1}{\sqrt{1-x^{2}}} $&
    $\displaystyle  \frac{d\arcsin x}{dx} $&
    $\displaystyle \int \frac{1}{\sqrt{1-x^{2}}}\, d x  $&
    $\displaystyle   \arcsin x+C $\\[3ex]
    $\displaystyle  f(x) + g(x) $&
    $\displaystyle  \frac{d F(x) + G(x)} {dx}$&
    $\displaystyle \int \{f(x)+g(x)\} \,dx $&
    $\displaystyle   F(x)+G(x)+C $&
    \\[3ex]
    $\displaystyle  cf(x) $&
    $\displaystyle  \frac{d\, cF(x)} {dx}$&
    $\displaystyle \int cf(x) \,dx $&
    $\displaystyle   cF(x)+C $&
    \\[3ex]
      $\displaystyle F\frac{dG}{dx}$&
      $\displaystyle \frac{dFG}{dx}-\frac{dF}{dx}G$&
      $\displaystyle \int FG'\,dx$&
      $\displaystyle FG-\int F'G\,dx$\\[3ex]
  \end{tabular}
  \smallskip
  \rule[1pt]{\textwidth}{0.3pt}

  To find derivatives and integrals involving $a^x$ instead of $e^x$ use
  $a = e^{\ln a}$,\\
  and thus $a^{x} = e^{x\ln a}$, to rewrite all exponentials as $e^{\ldots}$.
  \smallskip

  The following integral is also useful, but not as important as the
  ones above:
  \[
  \int \frac{ d x}{\cos x} = \frac12 \ln\frac{1+\sin x}{1-\sin x} +C
  \text{ for }\cos x\neq 0.
  \]
  \rule[6pt]{\textwidth}{2pt}
  \caption{The list of the standard integrals everyone should know}
  \label{tbl:standard-integrals}
\end{table}


%%% Local Variables: 
%%% mode: latex
%%% TeX-master: "free222"
%%% End: 
