\subsection{Inverse trig review}

\problem \groupproblem The \emph{inverse sine function} is the inverse
function to the (restricted) sine function, i.e. when $-\pi/2\le
\theta\le\pi/2$ we have
\[
\theta =\arcsin(y) \iff y=\sin\theta.
\]
The inverse sine function is sometimes called \emph{Arc Sine function}
and denoted $\theta=\arcsin(y)$.
We avoid the notation $\sin^{-1}(x)$ as it is ambiguous:
it could stand for either $\arcsin x$ or for $(\sin x)^{-1} = 1/(\sin x)$.


\subprob If $y=\sin\theta$, express $\sin\theta$, $\cos\theta$, and
$\tan\theta$ in terms of $y$ when \(0\le\theta<\pi/2\).

\subprob If $y=\sin\theta$, express $\sin\theta$, $\cos\theta$, and
$\tan\theta$ in terms of $y$ when $\pi/2<\theta\le\pi$.

\subprob If $y=\sin\theta$, express $\sin\theta$, $\cos\theta$, and
$\tan\theta$ in terms of $y$ when $-\pi/2<\theta<0$.

\subprob Evaluate $\DS\int\frac{\dd y}{\sqrt{1-y^2}}$ using the
substitution $y=\sin\theta$, but give the final answer in terms of
$y$.

\problem \groupproblem Express in simplest form:
\[
\tsubprob \cos(\arcsin^{-1}(x));\qquad
\tsubprob \tan \left\{ \arcsin\frac{\ln \frac14}{\ln 16}\right\};\qquad
\tsubprob \sin \bigl(2\arctan a\bigr)
\]

\problem \groupproblem  Draw the graph of $y=f(x) = \arcsin\bigl (\sin (x)\bigr)$,
for $-2\pi\leq x\leq +2\pi$. Make sure you get the same answer as your
graphing calculator.

\problem The \emph{inverse tangent function} is the inverse function to
the (restricted) tangent function, i.e. for $\pi/2< \theta<\pi/2$ we have
\[
\theta =\arctan(w) \iff w=\tan\theta.
\]
The inverse tangent function is sometimes called \emph{Arc Tangent  function}
and denoted $\theta=\arctan(y)$.
As with the arc sine, we avoid the notation $\tan^{-1}(x)$ since it
could stand for either $\arctan x$ or for $(\tan x)^{-1} = 1/(\tan x)$.

\subprob If $w=\tan\theta$, express $\sin\theta$ and
$\cos\theta$ in terms of $w$ when
\[
(a)\;\;  0\le\theta<\pi/2 \qquad
(b)\;\; \pi/2<\theta\le\pi \qquad
(c)\;\; -\pi/2<\theta<0
\]


\subsection{Integration by Parts and Reduction Formulae}


\problem Evaluate $\DS \int x^n \ln x\,\dd x$ where $n\ne-1$.

\answer
$\DS \int x^n \ln x\,\dd x
= \frac{x^{n+1}\ln x}{n+1} - \frac{x^{n+1}}{(n+1)^2}+C$.
\endanswer

\problem Evaluate $\DS \int e^{ax}\sin bx\,\dd x$ where $a^2+b^2\ne
0$.  [Hint: Integrate by parts twice.]

\answer
$\DS\int e^{ax}\sin bx\,\dd x= \frac{e^{ax}}{a^2+b^2}(a\sin bx
-b\cos bx)+C.$
\endanswer

\problem Evaluate $\DS\int e^{ax}\cos bx\,\dd x$ where $a^2+b^2\ne 0$.

\answer
$\DS\int e^{ax}\cos bx\,\dd x
= \frac{e^{ax}}{a^2+b^2}(a\cos bx+b\sin bx)+C.$
\endanswer



\problem Prove the formula
\[
\int x^n e^x\,\dd x= x^ne^x-n\int x^{n-1}e^x\,\dd x
\]
and use it to evaluate $\DS\int x^2e^x\,\dd x$.

\problem Prove the formula
\[
\int \sin^nx\,\dd x
= -\frac1n{\cos x\sin^{n-1}x}
+\frac{n-1}{n}\int\sin^{n-2}x\,\dd x,\qquad n\ne0
\]
\problem Evaluate $\DS\int\sin^2x\,\dd x$.  Show that the answer
is the same as the answer you get using the half angle formula.

\problem Evaluate $\DS\int_0^\pi\sin^{14}x\,\dd x$.
\answer
$\int_0^{\pi} \sin^{14}x dx = \frac{13\cdot11\cdot9\cdot7\cdot5\cdot3\cdot1 }
{14\cdot12\cdot10\cdot8\cdot6\cdot4\cdot2}\pi$
\endanswer

\problem Prove the formula
\[
\int \cos^nx\,\dd x
= \frac1n{\sin x\cos^{n-1}x}
+\frac{n-1}{n}\int\cos^{n-2}x\,\dd x,\qquad n\ne0
\]
and use it to evaluate $\DS\int_0^{\pi/4}\cos^4x\,\dd x$.
\answer
$\int \cos^n x dx = \frac1n\sin x \cos^{n-1}x
+\frac{n-1}n\int\cos^{n-2}x dx$;
$\int_0^{\pi/4}\cos^4 x dx = \frac7{16}+ \frac3{32}\pi$
\endanswer

\problem Prove the formula
\[
\int x^m(\ln x)^n\,\dd x= \frac{x^{m+1}(\ln
x)^n}{m+1}-\frac{n}{m+1}\int x^m(\ln x)^{n-1}\,\dd x,
\qquad m\ne -1,
\]
and use it to evaluate the following integrals:
\answer
Hint: first integrate $x^m$.
\endanswer
\problem $\DS\int \ln x\,\dd x$
\answer
$x\ln x-x+C$
\endanswer
\problem $\DS\int (\ln x)^2\,\dd x$
\answer
$x(\ln x)^2-2x\ln x+2x+C$
\endanswer
\problem $\DS\int x^3(\ln x)^2\,\dd x$
\problem Evaluate $\DS\int x^{-1}\ln x\,\dd x$ by another
method. [Hint: the solution is short!]
\answer
Substitute $u=\ln x$.
\endanswer

\problem For an integer $n > 1$ derive the formula
\[
\int \tan^n x\,\dd x=\frac 1{n-1}\tan^{n-1}x-\int \tan^{n-2}x\,\dd x
\]
Using this, find $\DS\int_0^{\pi/4} \tan^5x\,\dd x$ by doing just one
explicit integration.
\answer
$\int_0^{\pi/4}\tan^5 xdx =
\frac14(1)^4-\frac12(1)^2 + \int_0^{\pi/4}\tan x dx =
-\frac14+\ln\frac12\surd2$
\endanswer

Use the reduction formula from
example~\ref{ex:pt-fracs-reduction} to compute these integrals:
\problem $\DS \int \frac{\dd x}{(1+x^2)^3}$
\problem $\DS \int \frac{\dd x}{(1+x^2)^4}$
\problem $\DS \int \frac{x\dd x}{(1+x^2)^4}$
[Hint: $\int x/ (1+x^2)^n \dd x$ is easy.]
\problem $\DS \int \frac{1+ x}{(1+x^2)^2}\,\dd x$

\problem \groupproblem  The reduction formula from
example~\ref{ex:pt-fracs-reduction} is valid for all $n\neq0$. In
particular, $n$ does not have to be an integer, and it does not have
to be positive.

Find a relation between $\DS\int \sqrt{1+x^2}\,\dd x$ and
$\DS\int\frac{\dd x}{\sqrt{1+x^2}}$ by setting $n=-\frac12$.

\problem Apply integration by parts to \[\int \frac1x \;dx\]
Let
$u=\frac1x$ and $dv=dx$.  This gives us,\ $du=\frac{-1}{\;x^2}\,dx$
and $v=x$.
\[\int \frac1x \;dx = (\frac1x)(x) -\int x\;\frac{-1}{\;x^2}\;dx\]
Simplifying
\[\int \frac1x \;dx = 1 + \int \frac1x \;dx\]
and subtracting the integral from both sides gives us $0=1$.
How can this be?

\vskip 2ex


