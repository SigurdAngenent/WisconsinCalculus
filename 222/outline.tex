\def\chapter#1{\par{\bf #1}}
\def\section#1{\par\quad{\it #1}}
\def\subsection#1{\par\quad\quad{\rm #1}}
\def\label#1{ {\tt #1}}
\def\ref#1{ {\tt #1}}
\def\eqref#1{ {\tt #1}}
\def\textit#1{ {\it #1}}
\def\emph#1{ {\it #1}}

\chapter{Methods of Integration}
\section{The indefinite integral}
\subsection{Examples}
\section{You can always check the answer}
\subsection{Example}
\section{About ``$+C$''}
\section{Standard Integrals}
\section{Method of substitution}
\subsection{Example -- substitution in an indefinite integral}
\subsection{Example -- substitution in a definite integral}
\section{The double angle trick}
\subsection{Example}
\section{Problems}
\section{Integration by Parts}
\subsection{Example -- Integrating by parts once}
\subsection{Example -- Repeated Integration by Parts}
\subsection{Example -- Cousin Bruce's computation}
\section{Reduction Formulas}
\subsection{Example}
\subsection{Reduction formula requiring two partial integrations}
\subsection{A reduction formula where you have to solve for $I_n$}
\subsection{A reduction formula which will be handy later}
\section{Problems}
\section{Partial Fraction Expansion}
\subsection{Reduce to a proper rational function}
\subsection{Example}
\subsection{Partial Fraction Expansion: The Easy Case}
\subsection{Previous Example continued}
\subsection{Partial Fraction Expansion: The General Case}
\subsection{Example}
\section{Substitutions in integrals involving $\sqrt{ax^2+bx+c}$}
\subsection{Integrals involving $\sqrt{bx+c}$}
\subsection{Complete the square}
\section{Problems}
\chapter{Taylor's Formula}
\section{Taylor Polynomials}
\section{Examples}
\subsection{Example: Compute the Taylor polynomials of degree 0, 1 and 2 of}
\subsection{Example: Find the Taylor polynomials of $f (x)=\sin x$}
\subsection{Example -- Compute the Taylor polynomials of degree two and}
\section{Some special Taylor polynomials} \label{sec:some-special-taylor}
\section{The Remainder Term}
\subsection{Example}
\subsection{An unusual example, in which there \emph{is} a simple formula}
\subsection{Another unusual, but important example where you can compute}
\section{Lagrange's Formula for the Remainder Term}
\subsection{How to compute $e$ in a few decimal places}
\subsection{Error in the approximation $\sin x\approx x$}
\section{The limit as $x\to0$, keeping $n$ fixed }
\subsection{Little-oh}
\subsection{Example: prove one of these little-oh rules}
\subsection{Can you see that $x^3=o (x^2)$ by looking at the graphs of}
\subsection{Example: Little-oh arithmetic is a little funny}
\subsection{Computations with Taylor polynomials}
\subsection{How \textit{NOT} to compute the Taylor polynomial of degree 12}
\subsection{The right approach to finding the Taylor polynomial of}
\subsection{Example of multiplication of Taylor series}
\subsection{Taylor's formula and Fibonacci numbers}
\subsection{More about the Fibonacci numbers}
\subsection{Differentiating Taylor polynomials}
\subsection{Example: Taylor polynomial of $(1-x)^{-2}$}
\subsection{Example: Taylor polynomials of $\arctan x$. }
\section{Proof of Theorem \ref{thm:fisg-to-order-n}}
\section{Problems}
\subsection{TAYLOR'S FORMULA}
\subsection{Lagrange's Formula for the Remainder}
\subsection{Little-oh and Manipulating Taylor Polynomials}
\subsection{Limits of Sequences}
\subsection{Convergence of Taylor Series}
\subsection{Approximating Integrals}
\chapter{Differential Equations}
\section{What is a DiffEq?}
\section{First Order Separable Equations}
\subsection{Example}
\subsection{Example: The snag in action}
\section{First Order Linear Equations}
\subsection{The Integrating Factor}
\subsection{Variation of constants for 1st order equations}
\section{Dynamical Systems and Determinism}
\subsection{Example: Carbon Dating. }
\subsection{Example: On Dating a Leaky Bucket. }
\section{General Questions}
\section{Separation of Variables}
\section{Applications}
\chapter{Vectors}
\section{Introduction to vectors}
\subsection{Basic arithmetic of vectors}
\subsection{Some GOOD examples. }
\subsection{Two very, very BAD examples. }
\subsection{Algebraic properties of vector addition}
\subsection{Prove (\ref{eq:vector-addition-cmttve}). }
\subsection{Example}\label{ex:linear-combinations}
\section{Geometric description of vectors}
\subsection{Example. The point $P$ has coordinates $(2,3)$; the point $Q$}
\subsection{Example. Find the distance between the points $A$ and $B$ whose}
\subsection{Geometric interpretation of vector addition}
\subsection{Example}\label{ex:linear-combinations-picture}
\section{Parametric equations for lines and planes}
\subsection{Example} \textit{Find the parametric equation for the line}
\subsection{Midpoint of a line segment. }
\section{Vector Bases}
\subsection{The Standard Basis Vectors}
\subsection{A Basis of Vectors (in general)*}
\section{Dot Product}\label{sec:dot-product}
\subsection{Algebraic properties of the dot product}
\subsection{Example}
\subsection{The diagonals of a parallelogram}
\subsection{The dot product and the angle between two vectors}
\subsection{Orthogonal projection of one vector onto another}
\subsection{Defining equations of lines}
\subsection{Line through one point and perpendicular to another. }
\subsection{Distance to a line}
\subsection{Defining equation of a plane}
\subsection{Example} 
\subsection{Example continued}
\subsection{Where does the line through the points $B (2,0,0)$ and $C$}
\section{Cross Product}
\subsection{Algebraic definition of the cross product}
\subsection{Example}
\subsection{Example}
\subsection{Example}
\subsection{Algebraic properties of the cross product}
\subsection{The triple product and determinants}
\subsection{Geometric description of the cross product}
\section{A few applications of the cross product}
\subsection{Area of a parallelogram}
\subsection{Example}
\subsection{Finding the normal to a plane}
\subsection{Example}
\subsection{Volume of a parallelepiped}
\section{Notation}
\section{Problems}
\subsection{Computing and drawing vectors}
\subsection{Parametric equations for a line}
\subsection{Orthogonal decomposition of one vector with respect to another}
\subsection{The dot product}
\subsection{The cross product}
\vfill\eject\end
