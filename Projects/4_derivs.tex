% Time-stamp : Aug 22 15:59 4_derivs.tex

\project{Derivatives}
\centerline{(and the important topic of ants walking on
ropes)}

\section*{Purpose of this project}

In this project we will explore the following problem, which we will
solve using the rules of differentiation (and a little integration,
which will come later in the class).

\medskip

\subsection*{Problem statement:}\footnote{See also the wikipedia
  page on this problem at
  \url{http://en.wikipedia.org/wiki/Ant_on_a_rubber_rope}} Suppose
there is a rope that is very pliable stretched to be of size 1 meter.
Suppose that one end of the rope is pinned down and an ant is placed
at that end.  At that moment, the ant starts walking at a rate of $1$
meter per hour towards the other end.  However, at the same time we
begin stretching the rope (away from the pinned down side) so that the
end of the rope is moving at speed $10$ meters per second.  Will the
ant ever reach the end of the rope?

\medskip

While this version of the problem may seem strange, it actually has
relevance to a very reasonable question coming from cosmology: can the
light from a distant galaxy ever reach us if the universe is
expanding?  At issue is when a galaxy is so far away that the relative
speed between our galaxy and the other is greater than the speed of
light due to the expansion of the universe.  In this scenario, it
seems that the light from that far flung galaxy should never reach us.
However, in an analogous manner as the ant on the rubber rope, we can
be sure it does without having to resort to a breakdown in the laws of
physics.

\section{Solution}

We will solve the problem with the numbers given in the statement of
the problem.  The answer is ``yes, the ant does reach the end.''
Surprisingly, the fact that the ant eventually reaches the other end
does not depend upon the specific numbers given.  Thus, even if the
ant were walking at 0.0001 meters per second, and you were stretching
this rope near the speed of light, the ant will still eventually reach
the end.  Of course, the \textit{time} at which the ant reaches the
end depends upon the rates chosen, so it would be more accurate to say
that an immortal ant would eventually reach the end.

\bigskip

Any good solution to a hard word problem requires good notation!  So,
we begin with the following.
\begin{enumerate}
\item Let $D(t)$ denote the distance of the ant from its starting
  point at time $t$.  Thus, for example, $D(0) = 0$.
\item Note that because the end of the rope is moving at a constant
  speed of 10 meters per second, and because the end of the rope as
  located one meter from the pinned down edge at time zero, the
  position of the end of the rope at time $t$ is
  \[
  1 + 10t.
  \]
\item Let $P(t)$ denote the proportion of the rope that the ant has
  already traversed.  Note that until the ant reaches the end, $P(t)$
  is always between 0 and 1.  For example, if $P(t) = 0$, then the ant
  is at the beginning of the rope (which is true when $t=0$), and if
  $P(t) = 1$, then the ant has reached the end.  Also note that we can
  combine our knowledge of $D(t)$ and the position of the end of the
  rope to conclude that
  \[
  P(t) = \frac{D(t)}{1 + 10 t}.
  \]
\end{enumerate}

With the above notation in hand, we can start thinking about how to
solve the problem.  It should be clear that we hope to show that
\[
P(t^*) = 1,
\]
for some finite $t^*$.  However, to get there, we need to first
consider $D(t)$.

\section{The function $D(t)$}

We need to understand how the position of the ant, $D(t)$, is changing
with respect to time. %\bigskip
%
%\begin{exercise}
%  Argue that the ant is getting closer to the end of the rope when
%	 $D'(t) > 10$ and that it is getting further away when $D'(t)
%	 < 10$.
%\end{exercise}
%
The difficult issue is that the ant is moving due to two things
\begin{enumerate}
\item Its own movement relative to the rope.
\item The movement due to the pulling of the rope.
\end{enumerate}

Whenever we have more than one action forcing something else to change
we should try to break up the overall behavior into its components.
Therefore, we should expect to have

\[
D(t+h) = D(t) + \text{``movement due to ant only''} + \text{``movement
  due to rope''}.
\]

\subsection{Movement due to ant}
Since the ant is moving at a speed of $1$ m/h, and the amount of time
that passes is $h$ hours (which we think of as very small, of course),
we have
\[
D(t+h) = D(t) + 1 \cdot h + \text{``movement due to rope''}.
\]

\subsection{Movement due to rope}
The movement due to the rope seems tricker.  However, thinking along
the following lines will save the day:
\begin{enumerate}
\item We need to understand how fast a particular point on the rope is
  moving.
\item We should recognize that for each point on a rope, it's relative
  position on the rope (i.e. the proportion of the rope behind (or in
  front of) it) should not change with time.
\end{enumerate}

Thus, if $x\in [0,1]$ was a point on the rope at time zero, and $X(t)$
denotes its position at time $t$, then
\begin{equation}\label{eq:why?}
  \frac{x}{1} = \frac{X(t)}{1 + 10 t},
\end{equation}
for all $t$.  Since it holds for all $t$, we can conclude that
\begin{align}\label{eq:1}
  \frac{X(t+h)}{1 + 10(t+h)} = \frac{X(t)}{1 + 10 t},
\end{align}
since both sides equal the same thing (namely, $x/1$).

\section{Problem}
\problemfont
\problem
  Explain why equation \eqref{eq:why?} should hold for \textit{all} $t
  \ge 0$.  Show that equation \eqref{eq:1} implies that
  \begin{equation}\label{eq:from_rope}
    X(t + h) - X(t) = X(t) \frac{10 h}{1 + 10 t}.
  \end{equation}
\noproblemfont

\subsection{Returning to $D(t)$}
We return to our analysis of $D(t)$ armed with more knowledge.  Since
the position of the ant at time $t$ is $D(t)$, equation
\eqref{eq:from_rope} tells us that the total movement of the ant due
to the stretching rope over a time period of size $h$ is approximately
\[
D(t) \frac{10h}{1 + 10 t}.
\]
Combining all of the above, we have
\begin{align*}
  D(t + h) &= D(t) + \text{``movement due to ant only''} + \text{``movement due to rope''}\\
  &= D(t) + h + D(t) \frac{10h}{1 + 10 t}.
\end{align*}
Rearranging terms and dividing by $h$ gives
\[
\frac{D(t+h) - D(t)}{h} = 1 + D(t) \frac{10}{1 + 10 t}.
\]
Of course, we know that as $h \to 0$, the right hand side becomes
$D'(t)$.  Thus, we can conclude that
\begin{equation}\label{eq:ODE}
  D'(t) = 1 + D(t) \frac{10}{1 + 10t},
\end{equation}
where we also know that $D(0) = 0$.  The above equation tells us that
the function $D$ has the following properties:
\begin{enumerate}
\item It has a value of zero at time zero (we already knew that, of
  course!).
\item Its derivative is equal to itself multiplied by
  \[
  \frac{10}{1 + 10t},
  \]
  plus one.
\end{enumerate}

%\pagebreak
\section{So what?!?! -- Change perspective.}
Unfortunately, equation \eqref{eq:ODE} is a differential equation, and
is beyond the scope of this class to solve.  That is, at this point in
your mathematical education it is not clear at all that there should
even be a function satisfying such an odd equation.\footnote{There is
  one.  However, you will need more math courses to be able to solve
  these types of equations!}  Therefore, this seems like a dead-end.
%
%\begin{exercise}
%  Recall that the ant is getting closer to the end of the rope when
%	 $D'(t) > 10$ and is getting further away when $D'(t) <
%	 10$. Argue that if
%  \[
%		D(t) > \frac{9}{10}(1 + 10 t),
%  \]
%  which states that the ant is within 10\% of the end of the rope,
%	 then $D'(t) > 10$.
%\end{exercise}
%
%Of course, the above still does not show that the ant will get to the
% end of the rope!

However, at this point, we can remember that it was not $D(t)$ that we
were after anyways.  Instead, it was the function giving the
\textit{proportion} of the rope already traversed by time $t$,
\[
P(t) = \frac{D(t)}{1 + 10 t}.
\]

\section{Problem}
\problemfont
\problem Show that
\[
P'(t) = \frac{D'(t)}{1 + 10 t} - \frac{10 D(t)}{(1 + 10t)^2}.
\]
Then use \eqref{eq:ODE} to conclude that
\begin{equation}\label{eq:solve}
  P'(t) = \frac{1}{1 + 10 t}.
\end{equation}
\noproblemfont
\bigskip

It is important to note that equation \eqref{eq:solve} is much simpler
than equation \eqref{eq:ODE} since the right hand side does not depend
on anything except $t$.  Thus, equation \eqref{eq:solve} can be read
in the following manner:
\[
\text{``Find a function $P(t)$ whose derivative is} \ \ \frac{1}{1 +
  10 t}\text{.''}
\]
Note also that we require $P(0) = 0$.  Finding such a $P$ is actually
the exact \textit{opposite} of what you have learned to do so far,
which is how to \textit{take} a derivative.
 
 
 
\section{What to do?}

Later in the course, you will find an explicit representation for
$P(t)$ (if you must know now, the solution is $P(t) = (1/10) \ln(1 +
10 t)$, where ``$\ln$'' is the natural logarithm).  For now, it is
sufficient to know that there is a solution, and it yields a $t^*$ at
which
\[
P(t^*) = 1.
\]
For example, the value of $t^*$ for our specific choice of constants
is
\[
t^* = \frac{e^{10} - 1}{10} \approx 2,202.5.
\]


\section{Problems}
\problemfont
\problem Redo the analysis of the project except assume that the ant
is walking at a speed of $s$ meters per hour (as opposed to one meter
per hour) and that the end of the rope is being pulled at a rate of
$r$ meters per hour.  Of course, assume that $s>0$ and $r>0$.  You
should conclude that in this (general) case
\[
P'(t) = \frac{s}{1 + rt}.
\]


\problem Assume now that the ant is walking at 1 meter per hour and
the rope is being pulled at a rate of 2 meters per hour.  Using the
result from the previous exercise, argue that for small $h$,
\begin{equation}\label{eq:approx}
  P(t + h) \approx P(t) + \frac{1}{1 + 2t} h.
\end{equation}
Using this approximation for $P$, fill in the following tables to
approximate $t^*$ for which $P(t^*) = 1$:

\begin{minipage}[b]{0.3\textwidth}
  \begin{tabular}{cc}\toprule
    $t$ & $P(t)$ ($h = \frac12$)\\ \toprule
    0.0 & 0.00 \\ \midrule
    0.5 & 0.50  \\ \midrule
    1.0 & 0.75 \\ \midrule
    1.5 & 0.92 \\ \midrule
    2.0 & 1.04 \\ \midrule
    2.5 & 1.14\\ \bottomrule
  \end{tabular}
\end{minipage}
\begin{minipage}[b]{0.3\textwidth}
  \begin{tabular}{cc}\toprule
    $t$ & $P(t)$  ($h = \frac13$)\\ \toprule
    0.00 & 0.00 \\ \midrule
    0.333 & 0.33   \\ \midrule
    0.666 &  0.53 \\ \midrule
    1.000 &  \\ \midrule
    1.333 &  \\ \midrule
    1.666 &  \\ \midrule
    2.000 &  \\ \midrule
    2.666 &  \\ \midrule
    3.000 & \\ \midrule
    3.333 & \\ \bottomrule
  \end{tabular} 
\end{minipage}
\begin{minipage}[b]{0.3\textwidth}
  \begin{tabular}{cc}\toprule
    $t$ & $P(t)$  ( $h = \frac14$) \\ \toprule
    0.00 & 0.00 \\ \midrule
    0.25 &  0.25 \\ \midrule
    0.50 & 0.42 \\ \midrule
    0.75 &  \\ \midrule
    1.00 &  \\ \midrule
    1.25 &   \\ \midrule
    1.50 &  \\ \midrule
    1.75 &  \\ \midrule
    2.00 &  \\ \midrule
    2.25 &   \\ \midrule
    2.50 &  \\ \midrule
    2.75 &  \\ \midrule
    3.00 &  \\ \midrule
    3.25 & \\ \midrule
    3.50 & \\ \bottomrule
  \end{tabular}
\end{minipage}

For example, the first chart giving the computation with $h = 1/2$
yields an approximation
\[
t^* = 2.
\]


\noproblemfont


\section*{Report Instructions}
Using complete sentences,
write out solutions to all of the exercises.



