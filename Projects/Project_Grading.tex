\documentclass[12pt]{article}

\usepackage[left=1in,right=1in,top=1in,bottom=1in]{geometry}

\usepackage{enumerate,amsmath,amsfonts,amssymb,amsthm,url,framed}


\newtheorem{theorem}{Theorem}
\newtheorem{corollary}{Corollary}

% New environments
\theoremstyle{definition}
\newtheorem{example}{Example}
\newtheorem{definition}{Definition}
\newtheorem{remark}{Remark}
% New commands
\newcommand{\noin}{{\noindent}}
\def\R{\mathbb{R}}
\def\Z{\mathbb{Z}}
\def\E{\mathbb{E}}
\def\F{\mathcal{F}}
\def\FF{\mathcal{F}}
\newcommand\ind[1]{1_{\{#1\}}}


\begin{document}

\section*{Grading the projects}

\vspace{.1in}

\noindent Remember that the purpose of the projects are to 
\begin{enumerate}
\item Get the students to write something coherent in a mathematics
  course, and
\item Get them to think about deeper mathematical concepts than they
  are accustomed to.
\end{enumerate}

This semester, there are only going to be two projects given during
the course of the semester.  Thus, the projects will not play a large role in grading.  I think it will be best, therefore, that
the grading is ``light''.  That is, let's default to giving good
grades if a clear effort was made.  I recommend the following:

\begin{itemize}
\item Start with an assumption that the group has earned an $A$.
  Then, start taking points off according to the  rough rubric given below.
  
  \item   Below, I break the grading down into two categories: \textit{presentation}, and \textit{content}.
  \begin{enumerate}
  \item Presentation and appearance.
    \begin{itemize}
    \item If the appearance of the report is awful, take off up to one full letter
      grade.  Examples of a poor appearance/presentation
      include:
      \begin{enumerate}
      \item Project is not stapled.
      \item Paper has ``danglies'' as if it were just ripped out of a
        notebook.
        \item Papers looks like they were put into a blender before being turned in.
      \item etc.
      \end{enumerate}
      I'm sure there are a million other possibilities here.  The
      point is: if it does not look professional and care was given, then take some points off.
      \vspace{.2in}
	
    \item If it is clear that what they are turning in was their first
      attempt, take off a full letter grade.  Clear evidence includes:
      text crossed out, arrows pointing for you to read somewhere
      else, etc.  If they are too lazy to rewrite their solutions,
      they do not deserve a full grade.
    \end{itemize}
    
    \vspace{.2in}
	
  \item Content.
    \begin{itemize}
    \item If they have not tried to answer all the questions/exercises, take off points -- lots of points.  This is the meat of the assignment, and they need to at least attempt to solve everything.  If they can not solve something, they should say where they are stuck (this fact should be pointed out to the students). \vspace{.2in} 
    \item If they do not follow the instructions for the project writeup, take off points with the number of points (or size of deduction) depending on how much they left out. For example ``rehash what the project did and then discuss new material" has two instructions: 1) Rehash what we just did, and 2) discuss the new material.  Both should be done, and the new material should be put  in the context of material given to them.
    \end{itemize}
  \end{enumerate}
  
  \item \text{Note:}  Everything above depends upon presentation and making an effort in terms of content.  If they can not solve the mathematics being asked, but write about the efforts they made, I *do not* think they should lose more than one full letter grade.  There is plenty of time in this class to grade them on mathematical talent.  Here, you should be focussing on mathematical \textit{writing}.
\end{itemize}
\end{document}
