% Time-stamp : Sun May 27 19:39:04 CDT 2012


\project{Riemann Sums}

\section{Purpose of this project}

We will attempt to gain an understanding of the definite integral
\begin{equation*}
  \int_a^b f(x) dx,
\end{equation*}
by computing different types of Riemann sums.  We will learn that
there is more than one such sum, but that one of them is much more
accurate than the others.



\section{Approximations}

Consider the definite integral
\begin{equation*}
  \int_{0}^4 \sqrt{x+1} \;  dx.
\end{equation*}
We will use a variety of Riemann sums to approximate this integral.
First, we discretize $[0,4]$ into $N$ equally spaced intervals,
$[x_{k-1},x_{k}]$, with
\[
\Delta x = x_{k} - x_{k-1} = \frac{4}{N}.
\]
For example, in the case $N = 3$, we have $\{x_0,x_1,x_2,x_3\} =
\{0,\frac43,\frac83,4\}$ with associated intervals
\begin{equation*}
  [x_0,x_1] = [0,\tfrac43], \quad [x_1,x_2] = [\tfrac43,\tfrac83],
  \quad [x_2,x_3] = [\tfrac83,4].
\end{equation*}
Letting
\[
f(x) = \sqrt{x+1},
\]
we will now compute sums of the form
\begin{equation*}
  \sum_{k=1}^{N} f(c_k) \Delta x,
\end{equation*}
for different values of $N$, and where $c_k \in [x_{k-1},x_{k}]$ are the
\textit{sample points}.

\begin{center}
  See the lecture notes Chapter VII, Sections 1 and 2; also see the website
  \url{http://mathworld.wolfram.com/RiemannSum.html} for interactive pictures of
  Riemann sums.
\end{center}


\section{Left  sum}
For the ``Left sum'', we will choose $c_k = x_{k-1}$, i.e. the left endpoint of
the interval $[x_{k-1},x_{k}]$, for all $k = 1,\dots,N$.  For example, when $N =
3$, the left sum is
\begin{align*}
  f(0)\tfrac{4}{3} 
  +f({\tfrac{4}{3}}) \tfrac{4}{3} 
  +f({\tfrac{8}{3}}) \tfrac{4}{3} = 1\cdot
  \tfrac{4}{3} 
  +\sqrt{{\tfrac{7}{3}}}\; \tfrac{4}{3} 
  +\sqrt{{\tfrac{11}{3}}}\; \tfrac{4}{3} 
  \approx  5.9231,
\end{align*}
and when $N = 6$ the sum is
\begin{equation}
  \label{eq:righthand-sum3}
  \sum_{k = 1}^{6} f\left( \frac{4(k-1)}{6} \right) \frac{4}{6} \approx 6.3647.
\end{equation}
Using your calculator, software such as Excel, or the Riemann-sum website
\url{http://mathworld.wolfram.com/RiemannSum.html}, fill in the table below with
the associated left sums.

\begin{center}
  \begin{tabular}{cc}
    \toprule
    \rule[-8pt]{0pt}{14pt}
    $N$ & $\sum_{k = 1}^{N} f(x_{k-1})\Delta x$\\
    \toprule
    6 & 6.3647 \\ \midrule
    8 & \\ \midrule
    12 & \\ \midrule
    15 & \\ \bottomrule
  \end{tabular}
\end{center}


\section{Right sum}
For the ``Right sum'', we will choose $c_k = x_{k}$, i.e. the right endpoint of
the interval $[x_{k-1},x_{k}]$, for all $k = 1,\dots,N$.  For example, when $N =
3$, the right sum is
\begin{equation}
  \label{eq:lefthand-sum3}
  f({\tfrac{4}{3}}) \tfrac{4}{3} + f({\tfrac{8}{3}}) \tfrac{4}{3} +
  f({\tfrac{12}{3}})\tfrac{4}{3} = \sqrt{\tfrac{7}{3}}\tfrac{4}{3} +
  \sqrt{\tfrac{11}{3}}\tfrac{4}{3} + \sqrt{\tfrac{15}{3}} \tfrac{4}{3} \approx
  7.57126.
\end{equation}
Using your calculator, or software, such as Excel, fill in the table below with
the associated right sums.


\begin{center}
  \begin{tabular}{cc}
    \toprule
    \rule[-8pt]{0pt}{14pt}
    $N$ & $\sum_{k = 1}^{N} f(x_{k})\Delta x$\\
    \toprule
    6 & 7.18877 \\ \midrule
    8 & \\ \midrule
    12 & \\ \midrule
    15 & \\ \bottomrule
  \end{tabular}
\end{center}




\section{Midpoint sum}

The actual value of the integral is
\begin{equation}
  \label{eq:exact-value}
  \int_0^4 \sqrt{x+1} \;  dx = \frac{2} {3}\Bigl(5\sqrt{5} - 1\Bigr) \approx 6.786893.
\end{equation}
In the previous sections of this project you should have found that the left
sums always underestimated the correct solution for this problem while the right
sum always overestimated.  This should lead you to believe that perhaps it would
be best to choose $c_k$ to be the midpoint of $x_{k-1}$ and $x_{k}$.  That is, we
suspect that a good approximation method would be
\[
\sum_{k = 1}^{N} f\left(\frac{x_{k-1} + x_k}{2}\right) \Delta x.
\]
For example, when $N = 3$, the sum above sum is
\begin{align*}
  f({\tfrac{2}{3}}) \; \tfrac{4}{3} + f({\tfrac{6}{3}})\; \tfrac{4}{3} +
  f({\tfrac{10}{3}})\; \tfrac{4}{3} \approx 6.80628.
\end{align*}

Using your calculator, or Excel, fill in the table below with the associated
midpoint sums.


\begin{center}
  \begin{tabular}{cc}
    \toprule
    \rule[-8pt]{0pt}{14pt}
    $N$ & $\sum_{k = 1}^{N} f\left(\frac{x_{k-1} + x_k}{2}\right)\Delta x$\\
    \toprule
    6 & 6.79193 \\ \midrule
    8 & \\ \midrule
    12 & \\ \midrule
    15 & \\ \bottomrule
  \end{tabular}
\end{center}
% You should be able to compare the accuracy of the three methods discussed here
% (Left, Right, and Midpoint) and decide which one is ``converging'' to the
% correct answer much fast than the others.  In fact, if $E(1/N)$ is the error
% induced by the different approximation methods described here, then
% \begin{equation}\label{eq:tofind}
%   E_N =
%   \left| \int_a^b f(x) dx
%     - \sum_{k = 0}^{N-1} f\left( \frac{x_{k+1} + x_k}{2}\right) \right|
%   \approx  C \left(\frac{1}{N}\right)^p,
% \end{equation}
% where $C$ is some constant, and $p = 1$ for the left and right hand sums, and
% $p= 2$ for the midpoint sum.



\section{Report instructions}

Your project should include
\begin{enumerate}

\item Summary of ideas in project including all computations performed and
  conclusions reached. \textit{ Why did the left hand sum underestimate the true
  solution for this problem, while the right hand sum overestimated?}  (Hint:
  draw some pictures!)

\item Find formulas for $x_k$ and $c_k$ in the Left-sum
  \eqref{eq:righthand-sum3}.  What is $\frac{4(k-1)} {6}$ doing there?  Write
  the Right-sum \eqref{eq:lefthand-sum3} with $N=3$ using summation notation in
  the manner of Equation~\eqref{eq:righthand-sum3}.

\item Do the integral that leads to the actual value stated in
  Equation~\eqref{eq:exact-value}.

\item A function that is central to the study of probability is
  \begin{equation*}
    f(x) = e^{-x^2/2}.
  \end{equation*}
  Compute the left hand, right hand, and midpoint sums for this function
  integrated over the interval $[0,2]$ for several increasing values of $N$.
  \textit{Do the left hand sums underestimate the integral, or do they
    overestimate the integral?}  Stop each computation when you are confident
  that your $N$ is large enough so that your value is accurate to two decimal
  places.  How do you know you can stop when you did?
  
% \item Verify equation \eqref{eq:tofind}, for the Left, Right, and Midpoint sums
%   using the data found in this project.  Do this by completing the following
%   table:
  
\end{enumerate}
\bigskip
\begin{center}
  \begin{tabular}{r*{3}{p{96pt}}}
    \toprule\rule[-8pt]{0pt}{14pt}
    $N$ & \hfil left sum\hfil &\hfil right sum\hfil &\hfil midpoint sum \hfil \\
    \toprule
    6 & \\ \midrule
    8 & \\ \midrule
    12 & \\ \midrule
    15 & \\ \midrule
    $\ldots$ & \\ \midrule
    $\ldots$ & \\ \bottomrule
  \end{tabular}
\end{center}

% \begin{center}
%   \begin{tabular}{c*{6}{p{60pt}}}
%     \toprule\rule[-8pt]{0pt}{14pt}
%     $N$ & left sum & right sum & midpoint sum & $E_N$ & $NE_N$ & $N^2E_N$ \\
%     \toprule
%     6 & \\ \midrule
%     8 & \\ \midrule
%     12 & \\ \midrule
%     15 & \\ \bottomrule
%   \end{tabular}
% \end{center}
  


%%% Local Variables: 
%%% mode: latex
%%% TeX-master: "all_projects"
%%% End: 
