
\project{Limits}
\section{Purpose of this project}

We will explore how limits can be used to solve interesting problems,
some of which were ``paradoxes'' for thousands of years before
calculus came to the rescue.  In the next chapter of the notes, we
will see how limits can be used to formally define a notion of a
``derivative.''  In this project, we will use the limit for something
else: trying to figure out distances traveled and areas under curves
(which will later be called an \textit{integral}).  This project is
theoretical in nature.  However, students who put time into trying to
fully understand the concepts involved will surely be rewarded for
their efforts!

\section{Overview}

While the concept of the ``limit'' may seem hard to grasp, it is not
an exaggeration to say that it is \textit{the} conceptual breakthrough
that allowed for calculus to be developed (and, hence, all the nice
things that followed: air travel, mp3 players, etc.).  As we have
already seen, the limit is the necessary concept required to define a
derivative.  However, there is another object in calculus, the
\textit{integral}, that will be defined in the coming chapters and
that relies on the concept of a limit as well.  This project will
include our first computation of an integral, though we will not
formally make this connection until later in the semester.  We will
also point out how our computation of an integral is related to an
ancient paradox.

\section{A (seeming?) paradox}

The following is a version of a paradox usually attributed to the
Greek philosopher Zeno\footnote{See, for example,
  \url{http://en.wikipedia.org/wiki/Zeno's_paradoxes}.}: \vspace{.1in}

\noindent Imagine that a person is walking towards a wall that is
currently one mile away.  It seems that this person should have no
trouble walking to the wall.  However, we will consider a different
way to think about how much distance this person must travel to reach
the wall:
\begin{enumerate}%[(i.)]
\item The person must walk half to the wall, yielding a total distance
  of 1/2 mile.  This leaves 1/2 mile to go.
\item The person must walk one half of the remaining 1/2 mile.  This
  leaves 1/4 mile to go.
\item The person must walk one half of the remaining 1/4 mile.  This
  leaves 1/8 mile to go.
  \[\vdots\]
\end{enumerate}
Therefore, we have broken the interval of one mile down into an
infinite number of finite pieces of length
\[
1/2, \quad 1/4,\quad 1/8,\quad 1/16,\quad etc.,
\]
and so the total distance that must be traveled must be infinity!
Hence, \textit{the person should never reach the wall!}  Another way
to put this is that the requirement of walking a mile has been broken
into an infinite number of tasks (walking half of the remaining
distance over and over again), and this should be impossible.

\subsection{A resolution}

If the above argument does not sit well with you, then you have good
intuition.  Our resolution of the ``paradox'' rests on limits at
infinity (page 29 of the notes).  We will let $f(n)$ denote the
distance traveled by our wanderer after completion of the first $n$
``tasks'' as described in the section above.  Thus, we see
\begin{align*}
  f(1) &= \frac{1}{2}\\[1ex]
  f(2) &= \frac{1}{2} + \frac{1}{4}  = \frac{3}{4}\\[1ex]
  f(3) &=  \frac{1}{2} + \frac{1}{4}  + \frac{1}{8} = \frac{3}{4} + \frac{1}{8} = \frac{7}{8}\\[1ex]
  &\vdots\\[1ex]
  f(n) &= \frac{1}{2} + \frac{1}{4} + \cdots + \frac{1}{2^n} =
  \frac{2^n - 1}{2^n} = 1 - \frac{1}{2^n}.
\end{align*}
Therefore, using the Limit Property (P4) on page 39 of the text, with the
conclusion of Example 5.5 on page 37, we have that
\[
\lim_{n \to \infty} f(n) = \lim_{n\to \infty} \left(1 -
  \frac{1}{2^n}\right) = 1 - \lim_{n \to \infty} \frac{1}{2^n} = 1.
\]
Thus, even with the perspective of breaking the interval up into an
infinite number of intervals, we arrive at the (reasonable) conclusion
that the total distance traveled is one mile.

\section{Area under a triangle}

We now turn to a seemingly disparate topic: what is the area of a
triangle?  More specifically, consider the graph of the function $f(x)
= x$ on the interval $[0,1]$.  The space between the $x$-axis and
$f(x)$ for $x\in [0,1]$ makes a triangle and we want to know the area.
We will solve this problem in two ways.  The first is using simple
geometry, and will be familiar to everyone.  The second uses limits
and is a precursor to the much deeper idea of \textit{integration}
that you will be exposed to later in the course.

\vspace{.2in}

\noindent \textbf{Solution 1.}  The triangle has height equal to one,
and a base of size one.  Hence, the area of the triangle is
\[
\frac{1}{2} \cdot \text{base} \times \text{height} = \frac{1}{2}.
\]

\vspace{.1in}

\noindent \textbf{Solution 2.} Limits commonly arise when we wish to
make an approximation precise.  This is what we will do here.  More
precisely, we will make a series of approximations, with the $n$th
approximation denote by $A(n)$.  We will only assume that we know how
to compute the area of rectangle.

For $n \in \{1,2,3,\dots\}$, we will break the interval $[0,1]$ up
into $n$ equally spaced intervals.  That is,
\begin{enumerate}
\item For $n = 1$, there is only one interval, $[0,1]$.
\item For $n = 2$, there are two intervals, $[0,1/2]$ and $[1/2,1]$.
\item For $n = 3$, there are three intervals, $[0,1/3]$, $[1/3,2/3]$
  and $[2/3,1]$.
\item The $n$ intervals for an arbitrary integer $n$ are
  \[
  [0,1/n], [1/n,2/n], \dots, [(n-1)/n, 1].
  \]
\end{enumerate}
Now, we will approximate the area of the space between the graph of
$f(x)$ and the $x$-axis on each subinterval,
\[
[i/n,(i+1)/n],
\]
by the area of a rectangle, which we know how to compute.  Since $f(x)
= x$ for all $x$, we know that
\[
f(i/n) = i/n
\]
and we can approximate the area of the $i$th subinterval by
\[
f(i/n)\times \frac{1}{n} = \frac{i}{n} \frac{1}{n} = \frac{i}{n^2}.
\]
Adding up all these areas then yields the approximation (note that the
first interval is always being evaluated at the left endpoint, or
zero)
\[
A(n) = \sum_{i = 0}^{n-1} \frac{i}{n^2} = 0\times \frac{1}{n^2} +
1\times \frac{1}{n^2} + 2 \times \frac{1}{n^2} + \cdots + (n-1)
\frac{1}{n^2}.
\]
Note that $n$ is fixed in the above sum!

\section{Problems}
\problemfont
\problem Draw a detailed picture of the preceding argument.  More
specifically, for $n = 3, 4$, and $5$,

\subprob Plot $f(x) = x$ on the interval $[0,1]$.

\subprob Break the interval up into $n$ evenly spaced subintervals.

\subprob Show the area of the rectangles being computed.

\subprob Compute the approximate area for each $n\in \{3,4,5\}$.

\problem Let $m$ be an arbitrary integer.  We will compute
\[
1 + 2 + \cdots + m.
\]
To do so, let $S(m)$ be the (unknown) sum and add the following
vertically:
\[
\begin{array}[h]{c*{11}{@{}c}}
  &1& +& 2 &+& \cdots& + &(m-1)& +&  m &\;=\;& S(m)\\
 \rule[-6pt]{0pt}{14pt}
 & m& +& (m - 1)& +& \cdots& + &2& + &1 & = &S(m)\\
 \cline{2-12}
 \rule{0pt}{14pt}
  \implies& (m+1)& +& (m + 1)& +& \cdots&+& (m+1)& +& (m+1) &=& 2S(m)
\end{array}
\]
Conclude that
\[
S(m) = \frac{m(m+1)}{2}.
\]


\problem
Show that
\[
A(n) = \frac{n - 1}{2n} = \frac{1}{2} - \frac{1}{2n}.
\]
where $A(n)$ is the approximate to the area of the triangle after
breaking $[0,1]$ into $n$ subintervals.  Conclude that
\[
\lim_{n \to \infty} A(n) = \frac{1}{2}.
\]

\noproblemfont
\bigskip

\section*{Report Instructions}
Write a few paragraphs describing Zeno's paradox and its resolution.
In particular, note that the results from the class notes (page 29)
allowed us to conclude that
\[
\lim_{n \to \infty} \frac{1}{n} = 0,
\]
whereas we used that
\[
\lim_{n \to \infty} \frac{1}{2^n} = 0.
\]
Why is this okay to do?  Explain.

Write a few paragraphs describing how to compute the area of a
triangle.  Be sure to include solutions to all of the exercises.  Note
especially that the second method of computing the area is much harder
than the first.  However, what if the function we were approximating,
$f$, where not a straight line, but had a ``curve'' to it (like a sine
function)?  Say a few words on how one may hope to find the area
between its plot and the $x$-axis.  Note that you \textit{could not}
simply resort to using a straight geometric argument like we could for
the triangle.  We will formalize this method (and call it the
\textit{definite integral}) in later chapters.

