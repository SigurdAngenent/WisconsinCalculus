\documentclass[12pt]{article}

\usepackage[left=1in,right=1in,top=1in,bottom=1in]{geometry}

\usepackage{enumerate,amsmath,amsfonts,amssymb,amsthm}


\newtheorem{theorem}{Theorem}
\newtheorem{corollary}{Corollary}

% New environments
\theoremstyle{definition}
\newtheorem{example}{Example}
\newtheorem{definition}{Definition}
\newtheorem{remark}{Remark}
% New commands
\newcommand{\noin}{{\noindent}}
\def\R{\mathbb{R}}
\def\Z{\mathbb{Z}}
\def\E{\mathbb{E}}
\def\F{\mathcal{F}}
\def\FF{\mathcal{F}}
\newcommand\ind[1]{1_{\{#1\}}}


\begin{document}

\begin{itemize}
\item I think 6 projects is enough per course.  This leads to approximately one project every 2.5 weeks.

\item Some projects will be more application based than the lecture notes are.  I think this offers a very nice juxtaposition with the lecture material.   Some will be theoretical (integral).

\item The projects are essentially group exercises that are started in class, with the help of the TA.  Then, the groups, which should consist of 3 to 4 students, will meet outside of class to finish the project.

\item Added benefits:
\begin{enumerate}
	\item Learn more material and use what is learned in class.
	\item Some of the harder material should really peak the interest of even the top students.
	\item All students will get a study group out of it, which can be a huge benefit.
\end{enumerate}

\end{itemize}

\noindent Projects completed so far:
\begin{enumerate}
\item Chapter 1: functions.  Focus is on developing mathematical models of ``real world'' phenomenon.  There is room for more theoretic material for questions related to functions.  I also have the beginnings of a project where the students prove $\sqrt{2}$ and $\sqrt{3}$ are irrational.  But I'm not sure this is a good idea anymore.

	\item Chapter 2: Population growth models.  This project shows a few derivations of differential equations, though does not ask the students to \textit{solve} an ODE.  However, there is a nice question asking about the long term behavior of the solution to the logistic growth model.  This should get some students thinking.
	
	\item Chapter 3: Limits. Focus is on resolving Zeno's paradox and computing a first limit.  It's rather theoretical, but I like it.  We could also add a proof of the fact that the limit of a sum is the sum of the limits, but that seems disparate now.
	
	\item Chapter 4: Derivatives, Ants on a Rope!  We have an ant moving at speed $S_a$ along a rubber rope and the rubber rope expanding at a speed of $S_R$.  Think $S_R \gg S_a$.  Will the ant reach the end?  Physics question: can we eventually see any galaxy in an expanding universe?
	

	
	\item Chapter 6: Logarithms (log-log plots and semi-log plots).  Log-log for derivatives and and integrals.
	
	\item Chapter 7: The integral: approximating using left hand, right hand, and midpoint sums. Also uses some log-log.  This is more math theoretical.

\end{enumerate}

\end{document}
