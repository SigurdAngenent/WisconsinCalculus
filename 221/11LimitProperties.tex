%!TEX TS-program = xelatex
%Time-stamp: Aug  8 20:20 free221.tex
\documentclass{amsproc}
\input ../preamble
%\input ../gans.tex
\newcommand\version{1.0}
\newcommand\semester{Fall 2012}
%%%%%%%%%%%%%%%%%%%%%%%%%%%%%%%%%%%%%
%\includeonly{09intapps}
%%%%%%%%%%%%%%%%%%%%%%%%%%%%%%%%%%%%%
\begin{document}
\graphicspath{{figures/}}
\centerline{\bfseries MATH 221 -- The Limit properties}

\noindent\rule{\textwidth}{1pt}

\section{Special Limits}
\begin{align}
  \lim_{x\to a} c &= c\\
  \lim_{x\to a} x &= a\\
  \lim_{x\to a} \frac{\sin x}{x} &= 1
\end{align}

\noindent\rule{\textwidth}{1pt}

\section{Sums, Products and Quotients}
If $\lim_{x\to a} f(x) = A$ and $\lim_{x\to a}g(x)=B$ both exist, then
\begin{align}
  &\lim_{x\to a} f(x) +g(x) \text{ exists, and }&\lim_{x\to a} f(x) +g(x) &=A+B \\
  &\lim_{x\to a} f(x) -g(x) \text{ exists, and }&\lim_{x\to a} f(x) -g(x) &=A-B \\
  &\lim_{x\to a} f(x) g(x) \text{ exists, and }&\lim_{x\to a} f(x) g(x) &=AB \\
  &\lim_{x\to a} \frac{f(x) }{g(x)} \text{ exists, and }&\lim_{x\to a} \frac{f(x) }{g(x)}
  &=\frac{A}{B} \text{ provided }B\neq0.
\end{align}

\noindent\rule{\textwidth}{1pt}

\section{Inequalities}
If $\lim_{x\to a} f(x) = A$ and $\lim_{x\to a}g(x)=B$ both exist, and if $f(x)
\leq g(x)$ for all $x\neq a$,  then $A\leq B$, i.e.
\begin{equation}
  \lim_{x\to a} f(x) \leq \lim_{x\to a}g(x).
\end{equation}

If $\lim_{x\to a} f(x) $ and $\lim_{x\to a}g(x)$ both exist \emph{and are
equal}, and if there is a third function $h$ for which you know that
\[
  f(x) \leq h(x) \leq g(x) \text{ for all } x\neq a, 
\]
then, first of all, the limit $\lim_{x\to a} h(x)$ exists, and
\[
  \lim_{x\to a} f(x) = \lim_{x\to a} h(x) = \lim_{x\to a} g(x).
\]

\noindent\rule{\textwidth}{1pt}

\section{Substitution}
If $\lim_{x\to a} f(x) = A$ exists, and if the function $g(u)$ is continuous at
$u=A$ then
\[
  \lim_{x\to a} g\bigl(f(x)\bigr) = \lim_{u\to A} g(u) = g(A),
\]
or, written differently,
\begin{equation}
  \lim_{x\to a} g\bigl(f(x)\bigr) = g\bigl(\lim_{x\to a} f(x)\bigr).
\end{equation}



%\newpage
%%%%%%%%%%%%  GNU FDL  %%%%%%%%%%%%%%%
%%\input ../gfdl
\end{document}
