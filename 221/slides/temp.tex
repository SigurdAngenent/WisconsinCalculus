
\documentclass{amsart}
\usepackage{color}
\usepackage{graphicx}
\usepackage{nopageno}
\begin{document}
\setlength{\parindent}{0pt}
\pagestyle{empty}
\hrule
\begin{minipage}[b]{185pt}
  \input 03epsAndDelta073.tex
\end{minipage}
\begin{minipage}[b]{160pt}
  \sffamily%
  \color{blue}Conclusion:\\ this $L$ is not the limit of $f(x)$ as $x\to a$. 
  
  \rule{0pt}{18pt}
 \end{minipage}
\hrule
\begin{center}
    \sffamily\color[rgb]{0.75,0.0,0.0}The definition of ``$\lim_{x\to a} f(x) = L$'':\\
    \sffamily\color[rgb]{0.0,0.25,0.5} 
    For every $\varepsilon>0$ there is a $\delta>0$ such that \\
    for all $x\neq a$ with $a-\delta < x < a+\delta$ one has\\
    $L-\varepsilon < f(x) < L+\varepsilon$
\end{center}
\hrule
\end{document}
        