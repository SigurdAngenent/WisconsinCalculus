% Time-stamp: Aug 22 11:39 syllabus.tex
\documentclass{amsproc}
\begin{document}
\title{MIU-Math 221 Lecture schedule}
\maketitle
The following lecture schedule assumes there are 12 weeks in the semester,
whereas there are 14 weeks.  This allows time for in-class exams, and time to
catch up if one falls behind schedule.
\section*{Week 1}
\begin{itemize}
\item Chapter 1. ( things )
\item What is a number?
\item Functions
\item Implicit functions
\item Inverse functions
\item Inverse trigonometric functions
\end{itemize}
A big  difference between 221 and 275 (honors calculus) is that we do not try to
give a good description of the real numbers in 221 (no least upper bound axiom).
Instead we just assume that "everyone knows" what a real number is.  Here we
point out to the students that there are problems with this attitude, but that
we will ignore them this semester.  To show some of the problems, you can begin
by explaining that $\infty$ is not a number.  Then ask the class if they think
$0.9999\ldots =1$? (i.e.~if infinitely large numbers don't exist, do infinitely
small numbers such as $1-0.999\ldots$ exist?)

The rest of this chapter ought to be review.  I plan to explain/remind students of what a
function is, describe implicit and inverse functions, and review the arcsine and
arctangent.  The inverse trig functions are covered in a course which is a
prerequisite for math 221, so this should not be new material.  Still, students
may not know what the graphs of $y=\sin(\arcsin(x))$ and $y = \arcsin(\sin(x))$ look
like.


\section*{Week 2}
\begin{itemize}
\item Chapter 2. ( derivs )
\item The tangent to a curve
\item An example -- tangent to a parabola
\item Instantaneous velocity
\item Rates of change
\item Examples of rates of change
\end{itemize}
We introduce the derivative as a rate of change.  We don't get into the details
of how to compute derivatives, but this chapter motivates why we need to
understand limits.

\section*{Week 3}
\begin{itemize}
\item Chapter 3. ( limits )
\item Informal definition of limits
\item "The formal, authoritative, definition of limit"
\item Variations on the limit theme
\item Properties of the Limit
\item Examples of limit computations
\item When limits fail to exist
\item Limits that equal $\infty$
\item What's in a name? -- Free Variables and Dummy variables
\item Limits and Inequalities
\item Continuity
\item Substitution in Limits
\end{itemize}
There is a fair amount of text devoted to the $\varepsilon/\delta$ definition of
the limit.  Opinions on whether this is useful for the students and on how much
they will actually learn from this vary.  In other years I have put a fair
amount of time and energy into this topic, but this year I plan to spend only half a
(75 minute) lecture on the definition, and skip it otherwise. 
My current motives for this are that the $\varepsilon/\delta$ definition
provides an answer to a question most students won't think to ask.  It also
leads to exam questions that test the students' algebraic skills but not their
understanding of the $\varepsilon/\delta$ definition.  For instance, if you cover this
subject, then students should be able to answer the question ``show that
$\lim_{x\to 1} x^2 \neq 3$.''  I think that would be reasonable in math 275 or
necessary in math 421, but not here.

Instead of the $\varepsilon/\delta$ definition I do intend to use the limit
properties, and ask students to derive limits showing step by step how they use
those properties.

\section*{Week 4}
\begin{itemize}
\item Two Limits in Trigonometry
\item Asymptotes
\item Chapter 4. ( derivs )
\item Derivatives Defined
\item Direct computation of derivatives
\item Differentiable implies Continuous
\item Some non-differentiable functions
\end{itemize}
The trigonometric limits have a nice proof.  I will briefly describe what
asymptotes are (they should do the examples themselves for homework and/or in
discussion.)

The main example of direct computation of derivatives is $dx^n/dx$.  The
derivation in the text uses the geometric sum formula.  You could also use the
(first two terms of) the binomial formula.  

There are some examples of non differentiable functions.  I like to mention
Brownian motion as a continuous but nowhere differentiable function (no proof of
course) and I like to point out that this kind of ``pathological function''
actually occurs in physics (Brownian motion) but is also fundamental to
mathematical finance (there's a link to finance.yahoo.com for the prospective econ
students the class).

\section*{Week 5}
\begin{itemize}
\item The Differentiation Rules
\item Differentiating powers of functions
\item Higher Derivatives
\item Differentiating Trigonometric functions
\item The Chain Rule
\item Implicit differentiation
\end{itemize}
I would prove the product rule (straight forwardly, and also using the picture
proof.) For the quotient rule I would give the ``implicit differentiation
proof.''  It is not completely rigorous, but it has the advantage of being a
first example of implicit differentiation (while doing the proof I tell them
that this is an easier case of a method they should be able to use on an exam.)

For the derivative of trig functions I think I'll do the more geometric
derivation in problem 16 on page 72/73 this year.  It avoids the addition
formulas, and makes the answer perhaps less mysterious.   On the other hand it
would be nicer if the students did that problem themselves.

The ``related rate problems'' at the end are important because they force the
students to relate derivatives to the so-called real world, i.e. they test if
students can do more than ``use the formulas.''

\section*{Week 6}
\begin{itemize}
\item Chapter 5. ( graphsketching )
\item Tangent and Normal lines to a graph
\item The Intermediate Value Theorem
\item Finding sign changes of a function
\item Increasing and decreasing functions
\item Examples
\item Maxima and Minima
\item Must a function always have a maximum?
\item Examples -- functions with and without maxima or minima
\end{itemize}
Depending on how fast you go, and on how much attention you want to pay to the
Intermediate and Mean Value Theorems, this could actually be not enough material
for a whole week.
On the other hand, the example of the weird function on page 96 shows that
``$f'(x)>0$'' perhaps doesn't mean all that some people think it means.

\section*{Week 7}
\begin{itemize}
\item General method for sketching the graph of a function
\item Convexity, Concavity and the Second Derivative
\item Optimization Problems
\item Parametrized Curves
\end{itemize}
Optimization problems are again important to see if students know more than
``$dx^n/dx=nx^{n-1}$.''  To prove that the graph of a convex function always
lies below any of its chords you need the Mean value Theorem.  In math 275 I
would make this exam material.  Here I will explain it in lecture if there is
time.

The subject of parametrized curves can be shortened if necessary by skipping the
description of curvature and osculating circle.

\section*{Week 8}
\begin{itemize}
\item l'Hopital's rule
\item Chapter 7. ( expANDlog )
\item Exponents
\item Logarithms
\item Properties of logarithms
\item Graphs of exponential functions and logarithms
\item The derivative of $a^x$ and the definition of $e$
\item Derivatives of Logarithms
\item Limits involving exponentials and logarithms
\item Exponential growth and decay
\end{itemize}
The proof of l'Hopital's rule (or ``reason why it works'')
requires the MVT and parametrized curves.  A number of other departments have
told us they like to use l'Hopital, so it doesn't look good when we send them
students who don't know the rule.

There are many topics here, but the description of logarithms should be review.
The derivation of $de^x/dx=e^x$ is rigorous if you are willing to assume that
$a^x$ is defined for all $x\in\mathbb{R}$ and $a>0$, and if you are willing to
assume that at least $2^x$ is differentiable at $x=0$.

The topic of exponential growth and decay is important for many reasons:
exponential growth just shows up very often while $dX/dt = kX$ is the first
differential equation they run into. They will see more of them in 222.

\section*{Week 9}
\begin{itemize}
\item Chapter 8. ( integration )
\item Area under a Graph
\item When $f$ changes its sign
\item The Fundamental Theorem of Calculus
\item The summation notation
\item The indefinite integral
\item Properties of the Integral
\end{itemize}
The treatment here is a bit streamlined compared with more traditional
approaches to the integral that begin with long discourses on finding a formula
for $1^2+2^2+\cdots+n^2$.  The point of the fundamental theorem is that you
don't need to do that to find the area under a parabola.

\section*{Week 10}
\begin{itemize}
\item The definite integral as a function of its integration bounds
\item Method of substitution
\item Chapter 9. ( intapps )
\item Areas between graphs
\end{itemize}
This and the next two weeks seem straightforward.  Students should get a lot of
practice finding simple antiderivatives.
The point of the chapter on applications of integrals is to solidify the
students' intuition for the integral.  A complaint we have heard from
departments that use a lot of calculus is that their students know how to find
$\int (x^2+3x) dx$, but they have no idea of the intuitive meaning of the
integral as a sum of ``infinitely many infinitely small pieces.''

The formulas in the last chapter are less important than their intuitive
``derivation,''  where derivation does not mean rigorous mathematical proof, but
rather the simple intuitive arguments that physicists and engineers use to
reconstruct the formulas.

\section*{Week 11}
\begin{itemize}
\item Cavalieri's principle and volumes of solids
\item Three examples of volume computations of solids of revolution
\item Volumes by cylindrical shells
\item Distance from velocity
\end{itemize}

\section*{Week 12}
\begin{itemize}
\item The length of a curve
\item Velocity from acceleration
\item Work done by a force
\end{itemize}
\end{document}
