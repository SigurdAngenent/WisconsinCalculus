\section{Using limit properties to show a limit does \emph{not} exist}
The limit properties tell us how to prove that certain limits exist (and
how to compute them).  Although it is perhaps not so obvious at first
sight, they also allow you to prove that certain limits do not exist.  
This section shows a few examples of how to do this.  These examples
have one thing in common, namely they involve \emph{indirect reasoning},
or, \emph{proof by contradiction}.  This means that in all these examples
where we want to show that a certain limit $\lim_{x\to a} f(x)$ cannot
exist, we begin by saying ``well, suppose that the limit actually did
exist, and suppose it is $\lim_{xto a} f(x) = L$'';  we then use the
limit properties and after some reasoning arrive at a conclusion which
we know is blatantly false (e.g.~``$0=1$'').  
A correct reasoning which leads to a false conclusion must have
started from a false assumption.  The only assumption we made was that
$\lim_{x\to a} f(x)$ exists, and this assumption must therefore have
been wrong.


\subsection{Trying to divide by zero using a limit}%
\label{sec:xinverse-atzero-nolimit}%
The expression $1/0$ is not defined, but someone could try to give it
a meaning by computing
\[
\lim_{x\to0}\frac1x.
\]
After all that's what we did in Chapter 2, equation
\eqref{eq:tangent-slope-found}, when we computed the slope of the
tangent to a graph: we wanted to compute $\frac00$, and since $0/0$ is
not defined we replaced the fraction by a limit.  That worked and
allowed us to find the slope of the tangent to a parabola and later on
to other curves.  So now we ask: can you define $1/0$ by computing the
limit $\lim_{x\to0} 1/x$?

It turns out the answer is ``no'' because the limit does not exist.
Here are two reasons:

It is common wisdom that if you divide by a small number you get a large
number, so as $x\searrow 0$ the quotient $1/x$ will not be able to stay
close to any particular finite number, and the limit can't exist.

``Common wisdom'' is not always a reliable tool in mathematical proofs, so
here is a better argument.  The limit can't exist, because that would
contradict the limit properties $(P_1)\cdots(P_6)$.  Namely, suppose that
there were an number $L$ such that
\[
\lim_{x\to0} \frac 1 x = L.
\]
Then the limit property $(P_5)$ would imply that
\[
\lim_{x\to0}\bigl(\frac 1x\cdot x \bigr) = \bigl(\lim_{x\to0}\frac
1x\bigr)\cdot \bigl(\lim_{x\to0} x\bigr) = L\cdot 0 =0.
\]
On the other hand $\frac 1x \cdot x =1$ so the above limit should be 1!  A
number can't be both 0 and 1 at the same time, so we have a
contradiction. The assumption that $\lim_{x\to0}1/x$ exists is to blame, so
it must go.

\subsection{Another Example}
Does the limit
\begin{equation}
  \label{eq:03nolimit-bycontradiction}
  \lim_{x\to 0} \frac 1x-x
\end{equation}
exist? 

In \S~\ref{sec:xinverse-atzero-nolimit} we saw that the first term
($1/x$) has no limit at $x=0$. We know that the second term ($-x$) does have
a limit.  Therefore we could guess that the combination will not have
a limit:  ``the $1/x$ goes to $\pm\infty$, and the $x$ term is too
small to change that.''  To replace the guess by a watertight proof
using the limit properties, we argue by contradiction:

Let's suppose that what we want to prove is actually not true: in this
case, let's suppose that the limit in \eqref{eq:03nolimit-bycontradiction} does
exist.  

If you assume the limit \eqref{eq:03nolimit-bycontradiction} exists,
then, since we know $\lim_{x\to0}x=0$ exists, the limit properties
tell us that
\[
L=
\lim_{x\to 0} \bigl(\frac1x-x +x\bigr) 
=\lim_{x\to  0} \bigl(\frac1x-x\bigr) + \lim_{x\to 0} x
\]
should also exist.  But
\[
L=\lim_{x\to 0} \bigl(\frac1x-x +x\bigr) 
=\lim_{x\to  0} \frac1x
\]
and we know that this limit does \emph{not} exist
(\S\ref{sec:xinverse-atzero-nolimit} above).  
That's our contradiction:  we have a limit $L$ which both exists and does
not exist.  That  can't be, so our assumption in the beginning, namely
that the limit \eqref{eq:03nolimit-bycontradiction} exists, must be wrong.


\subsection{Limits at $\infty$ that don't exist}%
\label{ex:lim-at-infty-that-fails}%
If you let $x$ go to $\infty$, then $x$ will not get ``closer and closer''
to any particular number $L$, so it seems reasonable to guess that
\[
\lim_{x\to\infty}x\text{ does not exist.}
\]
You can prove this from the limit definition using an $\varepsilon$, and
there is also an easier indirect proof like the one we did above in
\S\ref{sec:xinverse-atzero-nolimit}.  You are
asked to provide it in exercise \ref{ex:limx-dne}.
Assuming that you've done exercise \ref{ex:limx-dne}, and that
we can now be sure that $\lim_{x\to\infty} x$ really does not exist,
we can also compute the limits of rational functions as $x\to\infty$
in the case we didn't do before in
\S\ref{sec:lim-at-infty-of-rationalfunction}.

Let's consider
\[
L=\lim_{x\to\infty} \frac{x^2+2x-1}{x+2}.
\]
Once again we factor the highest occurring powers of $x$ from
numerator and denominator,
\[
L=\lim_{x\to\infty}\frac{x^2}{x}\;\frac{1+2/x-1/x^2}{1+2/x}
=\lim_{x\to\infty}x\; \frac{1+2/x-1/x^2}{1+2/x}.
\]
The limit properties would let us write this as
\[
L=
\Bigl(\lim_{x\to\infty} \lim_{x\to\infty}x\Bigr) \times
\Bigl(\lim_{x\to\infty} \frac{1+2/x-1/x^2}{1+2/x}\Bigr)
\]
if only both limits existed.  But the first limit, $\lim_{x\to\infty}
x$ doesn't exist, so we're not allowed to use the limit properties in
this way.  We have to reason indirectly -- here goes:

Suppose the limit $L$ exists.  Then the following limit should also
exist:
\[
L\times \lim_{x\to\infty} \frac{1+2/x}{1+2/x-1/x^2}
=\lim_{x\to\infty}x\; \frac{1+2/x-1/x^2}{1+2/x} \frac{1+2/x}{1+2/x-1/x^2}.  
=\lim_{x\to\infty}x.
\]

