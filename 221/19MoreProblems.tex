%%%%%%%%%%%%%%%%%%%%%%%%%%%%%%%%%%%%%%%%%%%%%%%%
%!TEX root = free221.tex
%%%%%%%%%%%%%%%%%%%%%%%%%%%%%%%%%%%%%%%%%%%%%%%%
\chapter{More Problems}
\problemfont

For graphing problems you may be asked to determine

\noindent
(a)~where $f(x)$ is defined,\\
(b)~where $f(x)$ is continuous,\\
(c)~where $f(x)$ is differentiable,\\
(d)~where $f(x)$ is increasing and where it is decreasing,\\
(e)~where $f(x)$ is concave up and where it is concave down,\\
(f)~what the critical points of $f(x)$ are,\\
(g)~where the points of inflection are,\\
(h)~what  (if any) the horizontal asymptotes to $f(x)$ are, and\\
(i)~what  (if any) the vertical asymptotes to $f(x)$ are.\\

\noindent
(A horizontal line $y=b$ is called a {\em horizontal asymptote} if
$\ds\lim_{x\to\infty}f(x)=b$ or $\ds\lim_{x\to-\infty}f(x)=b$.  A vertical
line $x=a$ is called a {\em vertical asymptote} if $\ds\lim_{x\to
a+}f(x)=\pm\infty$ or $\ds\lim_{x\to a-}f(x)=\pm\infty$.)


For proofs the question will be carefully worded to indicate what you may
assume in your proof.  (See Problem~\ref{sinOver} for example.)  In this
document you may use without proof any previously asserted fact. For
example, you may use the fact that $\sin'(\theta)=\cos(\theta)$ to prove
that $\cos'(\theta)=-\sin(\theta)$ since the former question precedes the
latter below.  (See Problems~\ref{sineDerivative}
and~\ref{cosineDerivative}.)  You may always use high school algebra (like
$\cos(\theta)=\sin(\pi/2-\theta)$) in your proofs.

\medskip
\begin{multicols}{2}

  \problem State and prove the Sum Rule for derivatives.  You may use
  (without proof) the Limit Laws.


  \problem State and prove the Product Rule for derivatives.  You may use
  (without proof) the Limit Laws.


  \problem State and prove the Quotient Rule for derivatives.  You may use
  (without proof) the Limit Laws.


  \problem State and prove the Chain Rule for derivatives.  You may use
  (without proof) the Limit Laws.  You may assume (as the proof in the
  Stewart text does) that the inner function has a nonzero derivative.


  \problem State the Sandwich\footnote{Also called the Squeeze Theorem}
  Theorem.


  \problem Prove that $\ds \frac{d x^n }{ dx} = nx^{n-1}$, for all positive
  integers $n$.


  \problem Prove that $\ds \frac{d x^n }{ dx} = nx^{n-1}$, for $n=0$.


  \problem Prove that $\ds \frac{d x^n }{ dx} = nx^{n-1}$, for all negative
  integers $n$.


  \problem Prove that $\ds \frac{d e^x }{ dx} = e^x$.


  \problem\label{sinOver} Prove that
  \[
  \lim_{\theta\to 0} \frac{\sin(\theta)}{\theta} = 1.
  \]
  You may assume without proof the Sandwich Theorem, the Limit Laws, and that
  the $\sin$ and $\cos$ are continuous.  Hint: See Problem~\ref{SectorArea}.


  \problem Prove that
  \[
  \lim_{\theta\to 0} \frac{1-\cos(\theta)}{\theta} = 0.
  \]


  \problem\label{sineDerivative}
  Prove that $\ds \frac{d \sin x}{ dx} = \cos x$.


  \problem\label{cosineDerivative}
  Prove that $\ds \frac{d \cos x }{ dx} = -\sin x$.


  \problem Prove that $\ds\frac{d \tan x }{ dx} = \sec^2 x$.


  \problem Prove that $\ds\frac{d \cot x }{ dx} = -\csc^2 x$.


  \problem Prove that $\ds\frac{d \ln x }{ dx} = \frac{1}{x}$.


  \problem Prove that $\ds \frac{d \arcsin x}{ dx} = \frac{1}{
  \sqrt{1-x^2}}$.


  \problem Prove that $\ds \frac{d \arccos x }{ dx} = -\frac{1}{
  \sqrt{1-x^2}}$.


  \problem Prove that $\ds \frac{d \arctan x }{ dx} = \frac{1}{ 1+x^2}$.


  \problem True or false? A differentiable function must be continuous.  If
  true, give a proof; if false, illustrate with an example.


  \problem True or false? A continuous function must be differentiable.  If
  true, give a proof; if false, illustrate with an example.


  \problem Explain why $\ds \lim_{x\to 0} 1/x $ does not exist.


  \problem Explain why $\ds \lim_{\theta\to \pi/2} \tan \theta $ does not
  exist.


  \problem Explain why $\ds \lim_{x\to 0} \sin(1/x)$ does not exist.


  \problem Explain why $\ds \lim_{\theta\to \infty} \cos \theta $ does not
  exist.


  \problem Let $\sign(x)$ be the sign function (see
  example~\ref{ex:sign-function-has-no-limit}) Explain why $\ds\lim_{x\to 0}
  \sign(x) $ does not exist.


  \problem Explain why $\ds \lim_{y\to 0} 2^{1/y}$ does not exist.


  \problem Explain why $\ds \lim_{x\to 1} 2^{1/(x-1)} $ does not exist.


  \problem Calculate $\ds\lim_{\Delta x\to 0} \frac{f(x+\Delta x)-f(x)}{ (x+\Delta x)-x } $ when $f(x) = \sin 2x$.


  \problem Calculate $\ds\lim_{h\to 0} \frac{f(x+h)-f(x)}{ (x+h)-x }$ when $f(x) = \cos 2x$.


  \problem Calculate $\ds \lim_{x\to a} \frac{f(x)-f(a)}{ x-a } $ when $f(x) = \sin(x^2)$.


  \problem Calculate $\ds \lim_{x\to x_0} \frac{f(x)-f(x_0)}{x-x_0} $ when $f(x) = \cos(x^2)$.

  \hrule

  Calculate 
  \[
  \lim_{\Delta x\to 0} \frac{f(x+\Delta x)-f(x)}{ (x+\Delta x)-x }
  \]
  for the following functions :
  \problem $f(x)= \sqrt{\sin x}$.


  \problem  $f(x) = x \sin x$.


  \problem  $f(x) = e^{\sqrt{x}}$.


  \problem  $f(x) = e^{\sin x}$.


  \problem  $f(x) = \ln (ax+b)$.


  \problem  $f(x) = e^{\cos x}$.


  \problem  $f(x) =x^x$.


  \problem  $\ds f(x)= \frac{\sin x}{x}$.


  \problem  $f(x) = \sqrt{ax+b}$.


  \problem  $f(x) = (m x +c )^n$.

  \hrule


  \problem Use differentiation to estimate the number $\ds
  \frac{127^{4/3}-125^{4/3}}{2}$ approximately without a calculator. Your
  answer should have the form $p/q$ where $p$ and $q$ are integers.  Hint:
  $5^3=125$.


  \problem\label{CircleCircumference} What is the derivative of the area of a
  circle with respect to its radius?


  \problem \label{SphereArea} What is the derivative of the volume of a
  sphere with respect to its radius?


  \problem Find the slope of the tangent to the curve $y=x^3-x$ at $x=2$.


  \problem Find the equations of the tangent and normal to the curve
  $y=x^3-2x+7$ at the point $(1,6)$.


  \problem Find the equation of the tangent line to the curve $3xy^2-2x^2y=1$
  at the point $(1,1)$. Find $d^2y/dx^2$ at this point.


  \problem Find the equations of the tangent and normal to the curve
  $\ds\frac{x^2}{ a^2}+\frac{y^2}{ b^2}=1 $ at the point $(a\cos\theta,
  b\sin\theta)$.


  \problem Find the equations of the tangent and normal to the curve
  $\ds\frac{x^2}{ a^2}-\frac{y^2}{ b^2}=1 $ at the point $(a\sec\theta,
  b\tan\theta)$.


  \problem Find the equations of the tangent and normal to the curve
  $c^2(x^2+y^2)=x^2y^2$ at the point $(c/\cos\theta,c/\sin\theta)$.


  \problem Find the equations of the tangent and normal to the parabola $\ds
  y^2=4ax$ at the point $(at^2, 2at)$.


  \problem Show that the equation of the tangent to the hyperbola
  $\ds\frac{x^2}{ a^2}-\frac{y^2}{ b^2}=1 $ at the point $(p,q)$ is
  $\ds\frac{xp}{ a^2}-\frac{yq}{ b^2}=1 $


  \problem Find the equations of the tangent and normal to the curve
  $y=x^4-6x^3+13x^2-10x+5$ at the point where $x=1$.


  \problem Find the linear approximations to $\ds f(x)=\frac{1}{ \sqrt{4+x}}$
  at $x=0$.


  \problem Find the linear approximations to $f(x)=\sqrt{1+x}$ at $x=0$.


  \problem Find the linear approximations to $\ds f(x)=\frac{1}{ (1+2x)^4}$
  at $x=0$.


  \problem Find the linear approximations to $f(x)=(1+x)^3$ at $x=0$.


  \problem Find the linear approximations to $f(x)=\sec x$ at $x=0$.


  \problem Find the linear approximations to $f(x)=x\sin x$ at $x=0$.


  \problem Find the linear approximations to $f(x)=x^3$ at $x=1$.


  \problem Find the linear approximations to $f(x)=x^{1/3}$ at $x=-8$.


  \problem Find the linear approximations to $f(\theta)=\sin \theta$ at
  $\theta={\pi/6}$.


  \problem Find the linear approximations to $f(x)=x^{-1}$ at $x=4$.


  \problem Find the linear approximations to $f(x)=x^3-x$ at $x=1$.


  \problem Find the linear approximations to $f(x)=\sqrt{x}$ at $x=4$.


  \problem Find the linear approximations to $f(x)=\sqrt{x^2+9}$ at $x=-4$.


  \problem Use quadratic approximation to find the approximate value of
  $\sqrt{401}$ without a calculator.  Hint: $\sqrt{400}=20$.


  \problem Use quadratic approximation to find the approximate value of
  $(255)^{1/4}$ without a calculator. Hint: $256^{1/4}=4$.


  \problem Use quadratic approximation to find the approximate value of
  $\ds\frac{1}{ (2.002)^2} $ without a calculator.


  \problem Approximate $(1.97)^6$ without a calculator. (Leave arithmetic
  undone.)


  \problem Let $f$ be a function such that $f(1)=2$ and whose derivative is
  known to be $f'(x)=\sqrt{x^3+1}$.  Use a linear approximation to estimate
  the value of $f(1.1)$.  Use a quadratic approximation to estimate the value
  of $f(1.1)$.


  \problem Find the second derivative of $x^7$ with respect to $x$.


  \problem Find the second derivative of $\ln x$ with respect to $x$.


  \problem Find the second derivative of $5^x$ with respect to $x$.


  \problem Find the second derivative of $\tan \theta$ with respect to
  $\theta$.


  \problem Find the second derivative of $x^2e^{3x}$ with respect to $x$.


  \problem Find the second derivative of $\sin 3x\cos 5x$ with respect to
  $x$.


  \problem Find the third derivative of $u^4$ with respect to $u$.


  \problem Find the third derivative of $\ln x$ with respect to $x$.


  \problem Find the second derivative of $\tan x$ with respect to $x$.


  \problem If $\theta=\arcsin y$ show that $\ds\frac{d^2 \theta}{ dy^2} =
  \frac{y}{ (1-y^2)^{3/2} } $.


  \problem If $y=e^{-t}\cos t$ show that $\ds\frac{d^2 y}{ dt^2} =
  2e^{-t}\sin t$.


  \problem If $u=t+\cot t$ show that $\ds \sin^2 t\cdot \frac{d^2 u}{
  dt^2}-2u+2t=0 $.


  \problem If $y=e^{\tan x}$ show that $\ds \cos^2 x\cdot\frac{d^2 y}{
  dx^2}-(1+\sin 2x)\frac{dy}{dx}=0 $.


  \problem State L'H\^opital's rule and give an example which illustrates how
  it is used.


  \problem Explain why L'H\^opital's rule works.  Hint: Expand the numerator
  and the denominator in terms of $\Delta x$.


  \problem Give three examples to illustrate that a limit problem that looks
  like it is coming out to $0/0$ could be really getting closer and closer to
  almost anything and must be looked at a different way.


  \problem Give three examples to illustrate that a limit problem that looks
  like it is coming out to $1^\infty$ could be really getting closer and
  closer to almost anything and must be looked at a different way.


  \problem Give three examples to illustrate that a limit problem that looks
  like it is coming out to $0^0$ could be really getting closer and closer to
  almost anything and must be looked at a different way.


  \problem Give three examples to illustrate that a limit problem that looks
  like it is coming out to $\infty-\infty$ could be really getting closer and
  closer to almost anything and must be looked at a different way.


  \problem Explain how limit problems that come out to $\infty/\infty$ can
  always be converted into limit problems that come out to $0/0$ and why
  doing such a conversion is useful.


  \problem Explain how limit problems that come out to $\infty-\infty$ can be
  converted into limit problems that come out to $0/0$ and why doing such a
  conversion is useful.


  \problem Explain how limit problems that come out to $0^0$ can be converted
  into limit problems that come out to $0/0$ and why doing such a conversion
  is useful.


  \problem Explain how limit problems that come out to $1^\infty$ can be
  converted into limit problems that come out to $0/0$ and why doing such a
  conversion is useful.


  \problem\label{SectorArea} Use calculus to show that the area $A$ of a
  sector of a circle with central angle $\theta$ is $A=(\theta/2)R^2$ where
  $R$ is the radius and $\theta$ is measured in radians.  Hint: Divide the
  sector into $n$ equal sectors of central angle $\Delta\theta=\theta/n$ and
  area $\Delta A$.  As in the proof (see Problem~\ref{sinOver}) that
  \[
  \lim_{\Delta\theta\to 0} \frac{\sin(\Delta\theta)}{\Delta\theta} = 1,
  \]
  the area $\Delta A$ lies between the areas of two right triangles whose
  areas can be expressed in terms of $R$ and trig functions of
  $\Delta\theta$.  Apply the Sandwich Theorem to $A=n\Delta A$ and use
  l'H\^opital's rule or Problem~\ref{sinOver}.


  \problem\label{CircleArea} Use calculus to show that the area of a circle
  of radius $R$ is $\pi R^2$.  Hint: The area of a sector is a more general
  problem.  (See problem~\ref{SectorArea}.)


  \problem For which values of $x$ is the function $f(x)=x^2+3x+4$
  continuous?  Justify your answer with limits if necessary and draw a graph
  of the function to illustrate your answer.


  \problem For which values of $x$ is the function
  \[
  f(x)=
  \begin{cases}
    \frac{x^2-x-6}{x-3}, & \text{if } x\ne 3, \\
    5 &\text{if }x=3,
  \end{cases}
  \]
  continuous?  Justify your answer with limits if necessary and draw a graph
  of the function to illustrate your answer.


  \problem For which values of $x$ is the function
  \[
  f(x)=
  \begin{cases}
    \dfrac{\sin 3x}{x}, & \text{if } x\ne 0, \\
    1, & \text{if } x=0,
  \end{cases}
  \]
  continuous?  Justify your answer with limits if necessary and draw a graph
  of the function to illustrate your answer.


  \problem Determine the value of $k$ for which the function
  \[
  f(x)=
  \begin{cases}
    \dfrac{\sin 2x}{5x}& \text{if } x\ne 0,\\
    k, & \text{if } x=0,
  \end{cases}
  \]
  is continuous at $x=0$.  Justify your answer with limits if necessary and
  draw a graph of the function to illustrate your answer.


  \problem What does it mean for a function $f(x)$ to be continuous at $x=a$?


  \problem What does it mean for a function $f(x)$ to be differentiable at
  $x=a$?


  \problem What does $f'(a)$ indicate you about the graph of $y=f(x)$?
  Explain why this is true.


  \problem What does it mean for a function to be increasing?  Explain how to
  use calculus to tell if a function is increasing.  Explain why this works.


  \problem What does it mean for a function to be concave up?  Explain how to
  use calculus to tell if a function is concave up. Explain why this works.


  \problem What is a horizontal asymptote of a function $f(x)$?  Explain how
  to justify that a given line $y=b$ is a horizontal asymptote of $f(x)$.


  \problem What is a vertical asymptote of a function $f(x)$?  Explain how to
  justify that a given line $x=a$ is an vertical asymptote of $f(x)$.


  \problem If $f(x)=|x|$, what is $f'(-2)$?


  \problem Find the values of $a$ and $b$ so that the function
  \[
  f(x)=
  \begin{cases}
    x^2+3x+a,& \text{if } x\le 1,\\
    bx+2, & \text{if } x>1,
  \end{cases}
  \]
  is differentiable for all values of $x$.


  \problem Graph
  \[
  f(x)=
  \begin{cases}
    2-x,& \text{if } x\ge 1\\
    x, & \text{if } 0\le x\le 1.
  \end{cases}
  \]


  \problem Graph
  \[
  f(x)=
  \begin{cases}
    2+x, &\text{if } x\ge 0,\\
    2-x, &\text{if } x<0.
  \end{cases}
  \]


  \problem Graph
  \[
  f(x)=
  \begin{cases}
    1-x,& \text{if } x<1,\\
    x^2-1,&\text{if } x\ge 1.
  \end{cases}
  \]

  \problem Graph $\ds f(x) = x+1/x$.


  \problem Graph $\ds f(x)=\frac{x^2+2x-20}{ x-4} $ for $5<x<9$.


  \problem Graph $\ds f(x) = \frac{1}{ x^2+1} $.


  \problem Graph $f(x)=xe^x$.


  \problem State Rolle's theorem and draw a picture which illustrates the
  statement of the theorem.


  \problem State the Mean Value Theorem and draw a picture which illustrates
  the statement of the theorem.


  \problem Explain why Rolle's theorem is a {\it special case} of the Mean
  Value Theorem.


  \problem Let $f(x)=1-x^{2/3}$.  Show that $f(-1)=f(1)$ but that there is no
  number $c$ in the interval $(-1,1)$ such that $f'(c)=0$.  Why does this not
  contradict Rolle's theorem?


  \problem Let $f(x)=(x-1)^{-2}$.  Show that $f(0)=f(2)$ but that there is no
  number $c$ in the interval $(0,2)$ such that $f'(c)=0$.  Why does this not
  contradict Rolle's theorem?


  \problem Show that the Mean Value Theorem is not applicable to the function
  $f(x)=|x|$ in the interval $[-1,1]$.


  \problem Show that the Mean Value Theorem is not applicable to the function
  $f(x)=1/x$ in the interval $[-1,1]$.


  \problem Find a point on the curve $y=x^3$ where the tangent is parallel to
  the chord joining $(1,1)$ and $(3,27)$.


  \problem Show that the equation $x^5+10x+3=0$ has exactly one real root.


  \problem Find the local maxima and minima of $f(x)=(5x-1)^2+4$ without
  using derivatives.


  \problem Find the local maxima and minima of $f(x)=-(x-3)^2+9$ without
  using derivatives.


  \problem Find the local maxima and minima of $f(x)=-|x+4|+6$ without using
  derivatives.


  \problem Find the local maxima and minima of $f(x)=\sin 2x+5$ without using
  derivatives.


  \problem Find the local maxima and minima of $f(x)=|\sin 4x + 3|$ without
  using derivatives.


  \problem Find the local maxima and minima of $f(x)=x^4-62x^2+120x+9$.


  \problem Find the local maxima and minima of $f(x)=(x-1)(x+2)^2$.


  \problem Find the local maxima and minima of $f(x)=-(x-1)^3(x+1)^2$.


  \problem Find the local maxima and minima of $f(x)=x/2+2/x$ for $x>0$.

  \problem Find the local maxima and minima of $f(x)=2x^3-24x+107$ in the
  interval $[1,3]$.

  \problem Find the local maxima and minima of $f(x)=\sin x+(1/2)\cos x$ in
  $0\le x\le \pi/2$.




  \problem Show that the maximum value of $\ds \left(\frac{1}{ x}\right)^x$
  is $e^{1/e}$.




  \problem Show that $f(x)=x+1/x$ has a local maximum and a local minimum,
  but the value at the local maximum is less than the value at the local
  minimum.




  \problem Find the maximum profit that a company can make if the profit
  function is given by $p(x)=41+24x-18x^2$.




  \problem A train is moving along the curve $y=x^2+2$.  A girl is at the
  point $(3,2)$.  At what point will the train be at when the girl and the
  train are closest?  Hint: You will have to solve a cubic equation, but the
  numbers have been chosen so there is an obvious root.




  \problem Find the local maxima and minima of $f(x)=-x+2\sin x$ in
  $[0,2\pi]$.




  \problem Divide $15$ into two parts such that the square of one times the
  cube of the other is maximum.




  \problem Suppose the sum of two numbers is fixed.  Show that their product
  is maximum exactly when each one of them is half of the total sum.




  \problem Divide $a$ into two parts such that the $p$th power of one times
  the $q$th power of the other is maximum.




  \problem Which number between $0$ and $1$ exceeds its $p$th power by the
  maximum amount?




  \problem Find the dimensions of the rectangle of area $96$ cm${}^2$ which
  has minimum perimeter.  What is this minimum perimeter?




  \problem Show that the right circular cone with a given volume and minimum
  surface area has altitude equal to $\sqrt{2}$ times the radius of the base.




  \problem Show that the altitude of the right circular cone with maximum
  volume that can be inscribed in a sphere of radius $R$ is $4R/3$.




  \problem Show that the height of a right circular cylinder with maximum
  volume that can be inscribed in a given right circular cone of height $h$
  is $h/3$.




  \problem A cylindrical can is to be made to hold 1 liter of oil.  Find the
  dimensions of the can which will minimize the cost of the metal to make the
  can.




  \problem An open box is to be made out of a given quantity of cardboard of
  area $p^2$.  Find the maximum volume of the box if its base is square.





  \problem Find the dimensions of the maximum rectangular area that can be
  fenced with a fence 300 yards long.




  \problem Show that the triangle of the greatest area with given base and
  vertical angle is isosceles.




  \problem Show that a right triangle with a given perimeter has greatest
  area when it is isosceles.



  \problem What do distance, speed and acceleration have to do with calculus?
  Explain thoroughly.




  \problem A particle, starting from a fixed point $P$, moves in a straight
  line.  Its position relative to $P$ after $t$ seconds is $s=11+5t+t^3$
  meters.  Find the distance, velocity and acceleration of the particle after
  $4$ seconds, and find the distance it travels during the $4$th second.




  \problem The displacement of a particle at time $t$ is given by
  $x=2t^3-5t^2+4t+3$.  Find (i)~the time when the acceleration is $8{\rm
  cm/s}^2$, and (ii)~the velocity and displacement at that instant.




  \problem A particle moves along a straight line according to the law
  $s=t^3-6t^2+19t-4$.  Find (i)~its displacement and acceleration when its
  velocity is $7$m/s, and (ii)~its displacement and velocity when its
  acceleration is $6{\rm m/s}^2$.




  \problem A particle moves along a straight line so that after $t$ seconds
  its position relative to a fixed point $P$ on the line is $s$ meters, where
  $s=t^3-4t^2+3t$.  Find (i)~when the particle is at $P$, and (ii)~its
  velocity and acceleration at these times $t$.




  \problem A particle moves along a straight line according to the law
  $s=at^2-2bt+c$, where $a,b,c$ are constants.  Prove that the acceleration
  of the particle is constant.




  \problem The displacement of a particle moving in a straight line is
  $x=2t^3-9t^2+12t+1$ meters at time $t$.  Find (i)~the velocity and
  acceleration at $t=1$ second, (ii)~the time when the particle stops
  momentarily, and (iii)~the distance between two stops.




  \problem The distance $s$ in meters travelled by a particle in $t$ seconds
  is given by $s=ae^t+be^{-t}$.  Show that the acceleration of the particle
  at time $t$ is equal to the distance the particle travels in $t$ seconds.



  \problem A ladder $10$ feet long rests against a vertical wall.  If the
  bottom of the ladder slides away from the wall at a speed of $2$ ft/s, how
  fast is the angle between the top of the ladder and the wall changing when
  the angle is $\pi/4$ radians?




  \problem A ladder $13$ meters long is leaning against a wall.  The bottom
  of the ladder is pulled along the ground away from the wall at the rate of
  $2$ m/s.  How fast is its height on the wall decreasing when the foot of
  the ladder is $5$ m away from the wall?




  \problem A television camera is positioned $4000$ ft from the base of a
  rocket launching pad.  A rocket rises vertically and its speed is $600$
  ft/s when it has risen $3000$ feet.  (a)~How fast is the distance from the
  television camera to the rocket changing at that moment?  (b)~How fast is
  the camera's angle of elevation changing at that same moment?  (Assume that
  the telivision camera points toward the rocket.)




  \problem Explain why exponential functions arise in computing radioactive
  decay.



  \problem Explain why exponential functions are used as models for
  population growth.



  \problem Radiocarbon dating works on the principle that $^{14}C$ decays
  according to radioactive decay with a half life of $5730$ years.  A
  parchment fragment was discovered that had about $74\%$ as much $^{14}C$ as
  does plant material on earth today.  Estimate the age of the parchment.


  \problem If $f'(x)=x-1/x^2$ and $f(1)=1/2$ find $f(x)$.


  \problem $\ds \int (6x^5-2x^{-4}-7x+3/x-5+4e^x+7^x) \,dx$


  \problem $\ds \int (x/a+a/x+x^a+a^x+ax) \,dx $


  \problem $\ds\int \left(\sqrt{x}- \root 3\of {x^4}+\frac{7}{ {\root 3 \of
  {x^2}}} -6e^x+1\right)\,dx $


  \problem$\ds\int 2^x \,dx$


  \problem$\ds\int^4_{-2} (3x-5) \,dx$


  \problem$\ds \int^2_{1} x^{-2} \,dx$


  \problem$\ds \int^1_{0} (1-2x-3x^2) \,dx$


  \problem$\ds \int^2_1 (5x^2-4x+3) \,dx$


  \problem$\ds \int^0_{-3} (5y^4-6y^2+14) \,dy$


  \problem$\ds \int^1_{0} (y^9-2y^5+3y) \,dy$


  \problem$\ds \int^4_0 \sqrt{x} \,dx$


  \problem$\ds \int^1_0 x^{3/7} \,dx$


  \problem$\ds \int^3_1 \left(\frac{1}{ t^2}-\frac{1}{ t^4}\right) \,dt$


  \problem$\ds \int^2_1 \frac{t^6-t^2}{ t^4} \,dt$


  \problem$\ds \int^2_1 \frac{x^2+1}{ \sqrt{x}} \,dx$


  \problem$\ds \int^2_0 (x^3-1)^2 \,dx$


  \problem$\ds \int^1_0 u(\sqrt{u}+{\root 3 \of u}) \,du$


  \problem$\ds \int^2_{1} (x+1/x)^2 \,dx$


  \problem$\ds \int^3_{3} \sqrt{x^5+2} \,dx$


  \problem$\ds \int^{-1}_1 (x-1)(3x+2) \,dx$


  \problem$\ds \int^{4}_1 (\sqrt{t}-2/\sqrt{t}) \,dt$


  \problem$\ds \int^{8}_1 \left({\root 3\of r}+\frac{1}{ {\root 3 \of
  r}}\right) \,dr$


  \problem$\ds \int^{0}_{-1} (x+1)^3 \,dx$


  \problem$\ds \int^{-2}_{-5} \frac{x^4-1}{ x^2+1} \,dx$


  \problem$\ds \int^e_{1} \frac{x^2+x+1}{ x} \,dx$


  \problem$\ds \int^9_{4} \left(\sqrt{x}+\frac{1}{ \sqrt{x}}\right)^2 \,dx$


  \problem$\ds \int^1_{0} \left({\root 4\of {x^5}}+{\root 5\of {x^4}}\right)
  \,dx$


  \problem$\ds \int^8_{1} \frac{x-1}{ {\root 3\of {x^2}}} \,dx$


  \problem$\ds \int^{\pi/3}_{\pi/4} \sin t \,dt$


  \problem$\ds \int^{\pi/2}_0 (\cos \theta + 2\sin \theta) \,d\theta$


  \problem$\ds \int^{\pi/2}_0 (\cos \theta + \sin 2\theta) \,d\theta$


  \problem$\ds \int^{\pi}_{2\pi/3} \sec x\tan x \,dx$


  \problem$\ds \int^{\pi/2}_{\pi/3} \csc x\cot x \,dx$


  \problem$\ds \int^{\pi/3}_{\pi/6} \csc^2 \theta \,d\theta$


  \problem$\ds \int^{\pi/3}_{\pi/4} \sec^2 \theta \,d\theta$


  \problem$\ds \int^{\sqrt{3}}_{1} \frac{6}{ {1+x^2}} \,dx$


  \problem$\ds \int^{0.5}_{0} \frac{dx}{ \sqrt{1-x^2}}$


  \problem$\ds \int^{8}_{4} (1/x) \,dx$


  \problem$\ds \int^{\ln 6}_{\ln 3} 8e^x \,dx$


  \problem$\ds \int^{9}_{8} 2^t \,dt$


  \problem$\ds \int^{-e}_{-e^2} \frac{3}{ x} \,dx$


  \problem$\ds \int^3_{-2} |x^2-1| \,dx$


  \problem$\ds \int^2_{-1} |x-x^2| \,dx$


  \problem$\ds \int^2_{-1} (x-2|x|) \,dx$


  \problem$\ds \int^2_{0} (x^2-|x-1|) \,dx$




  \problem $\ds \int^2_{0} f(x) \,dx$ where
  \[
  f(x) =
  \begin{cases}
    x^4,  &\text{if } 0\le x<1,\\
    x^5, &\text{if } 1\le x\le 2.
  \end{cases}
  \]




  \problem $\ds \int^\pi_{-\pi} f(x) \,dx$ where
  \[
  f(x) =
  \begin{cases}
    x,&\text{if } -\pi\le x\le 0,\\
    \sin x, & \text{if } 0< x\le \pi.
  \end{cases}
  \]




  \problem True or false?  $\ds\int^1_{-1} \frac{3}{t^4}
  \,dt=\left.\frac{-1}{t^3}\right|_{-1}^1=-1+1=0$.



  \problem Explain what a Riemann sum is and write the definition of $\ds
  \int_a^b f(x) dx $ as a limit of Riemann sums.





  \problem State the Fundamental Theorem of Calculus.



  \problem (1)~Water flows into a container at a rate of three gallons per
  minute for two minutes, five gallons per minute for seven minutes and
  eleven gallons per minute for two minutes.  How much water is in the
  container?  (2)~Water flows into a container at a rate of $t^2$ gallons per
  minute for $0\le t\le 5$.  How much water is in the container?




  \problem Let $f(x)$ be a function which is continuous and let $A(x)$ be the
  area under $f(x)$ from $a$ to $x$.  Compute the derivative of $A(x)$ by
  using limits.




  \problem Why is the Fundamental Theorem of Calculus true?  Explain
  carefully and thoroughly.




  \problem Give an example which illustrates the Fundamental Theorem of
  Calculus.  In order to do this compute an area by summing up the areas of
  narrow rectangles and then show that applying the Fundamental Theorem of
  Calculus gives the same answer.





  \problem Sketch the graph of the curve $y=\sqrt{x+1}$ and determine the
  area of the region enclosed by the curve, the $x$-axis and the lines $x=0$,
  $x=4$.


  \problem Make a sketch of the graph of the function $y=4-x^2$ and determine
  the area enclosed by the curve, the $x$-axis and the lines $x=0$, $x=2$.


  \problem Find the area under the curve $y=\sqrt{6x+4}$ and above the
  $x$-axis between $x=0$ and $x=2$.  Draw a sketch of the curve.


  \problem Graph the curve $y=x^3$ and determine the area enclosed by the
  curve and the lines $y=0$, $x=2$ and $x=4$.




  \problem Graph the function $f(x)=9-x^2$, $0\le x\le 3$, and determine the
  area enclosed between the curve and the $x$-axis.


  \problem Graph the curve $y=2\sqrt{1-x^2}$, $x\in [0,1]$, and find the area
  enclosed between the curve and the $x$-axis.


  \problem Determine the area under the curve $y=\sqrt{a^2-x^2}$ and between
  the lines $x=0$ and $x=a$.


  \problem Graph the curve $y=2\sqrt{9-x^2}$ and determine the area enclosed
  between the curve and the $x$-axis.


  \problem Graph the area between the curve $y^2=4x$ and the line $x=3$.
  Find the area of this region.


  \problem Find the area bounded by the curve $y=4-x^2$ and the lines $y=0$
  and $y=3$.


  \problem Find the area bounded by the curve $y=x(x-3)(x-5)$, the $x$-axis
  and the lines $x=0$ and $x=5$.


  \problem Find the area enclosed between the curve $y=\sin 2x$, $0\le x\le
  \pi/4$ and the axes.


  \problem Find the area enclosed between the curve $y=\cos 2x$, $0\le x\le
  \pi/4$ and the axes.


  \problem Find the area enclosed between the curve $y=3\cos x$, $0\le x\le
  \pi/2$ and the axes.


  \problem Find the area enclosed between the curve $y=\cos 3x$, $0\le x\le
  \pi/6$ and the axes.


  \problem Find the area enclosed between the curve $y=\tan^2 x$, $0\le x\le
  \pi/4$ and the axes.


  \problem Find the area enclosed between the curve $y=\csc^2 x$, $\pi/4\le
  x\le \pi/2$ and the axes.


  \problem Find the area of the region bounded by $y=-1$, $y=2$, $x=y^3$, and
  $x=0$.


  \problem Find the area of the region bounded by the parabola $y=4x^2$,
  $x\ge 0$, the $y$-axis, and the lines $y=1$ and $y=4$.


  \problem Find the area of the region bounded by the curve $y=4-x^2$ and the
  lines $y=0$ and $y=3$.


  \problem Graph $y^2+1=x$, $x\le 2$ and find the area enclosed by the curve
  and the line $x=2$.


  \problem Graph the curve $y=x/\pi+2\sin^2 x$ and write a definite integral
  whose value is the area between the $x$-axis, the curve and the lines $x=0$
  and $x=\pi$.  {\em Do not} evaluate the integral.  {\em Do} specify the
  limits of integration.




  \problem Find the area bounded by $y=\sin x$ and the $x$-axis between $x=0$
  and $x=2\pi$.  Hint: Make a careful drawing to decide what area is
  intended.


  \problem Find the area bounded by the curve $y=\cos x$ and the $x$-axis
  between $x=0$ and $x=2\pi$.


  \problem Give an example which shows that $\ds\int_a^b f(x)\, dx$ is not
  always the true area bounded by the curves $y=f(x)$, $y=0$, $x=a$, and
  $x=b$ even though $f(x)$ is continuous between $a$ and $b$.


  \problem Find the area of the region bounded by the parabola $y^2=4x$ and
  the line $y=2x$.


  \problem Find the area bounded by the curve $y=x(2-x)$ and the line $x=2y$.


  \problem Find the area bounded by the curve $x^2=4y$ and the line $x=4y-2$.


  \problem Calculate the area of the region bounded by the parabolas $y=x^2$
  and $x=y^2$.


  \problem Find the area of the region included between the parabola $y^2=x$
  and the line $x+y=2$.


  \problem Find the area of the region bounded by the curves $y=\sqrt{x}$ and
  $y=x$.


  \problem Use integration to find the area of the triangular region bounded
  by the lines $y=2x+1$, $y=3x+1$ and $x=4$.


  \problem Find the area bounded by the parabola $x^2-2=y$ and the line
  $x+y=0$.


  \problem Graph the curve $y=(1/2)x^2+1$ and the straight line $y=x+1$ and
  find the area between the curve and the line.


  \problem Find the area of the region between the parabolas $y^2=x$ and
  $x^2=16y$.


  \problem Find the area of the region enclosed by the parabola $y^2=4ax$ and
  the line $y=mx$.


  \problem Find $a$ so that the curves $y=x^2$ and $y=a\cos x$ intersect at
  the points $(x,y)=(\frac\pi4, \frac{\pi^2}{16})$. Then find the area
  between these curves.


  \problem Write a definite integral whose value is the area of the region
  between the two circles $x^2+y^2=1$ and $(x-1)^2+y^2=1$.  Find this area.
  If you cannot evaluate the integral by calculus you may use geometry to
  find the area.  Hint: The part of a circle cut off by a line is a circular
  sector with a triangle removed.


  \problem Write a definite integral whose value is the area of the region
  between the circles $x^2+y^2=4$ and $(x-2)^2+y^2=4$.  {\em Do not} evaluate
  the integral.  {\em Do} specify the limits of integration.


  \problem Write a definite integral whose value is the area of the region
  between the curves $x^2+y^2=2$ and $x=y^2$ {\em Do not} evaluate the
  integral.  {\em Do} specify the limits of integration.


  \problem Write a definite integral whose value is the area of the region
  between the curves $x^2+y^2=2$ and $x=y^2$.  Find this area.  If you cannot
  evaluate the integral by calculus you may use geometry to find the area.
  Hint: Divide the region into two parts.


  \problem Write a definite integral whose value is the area of the part of
  the first quadrant which is between the parabola $y^2=x$ and the circle
  $x^2+y^2-2x=0$.  Find this area.  If you cannot evaluate the integral by
  calculus you may use geometry to find the area.  Hint: Draw a careful
  graph. Divide a semicircle in two.


  \problem Find the area bounded by the curves $y=x$ and $y=x^3$.


  \problem Graph $y=\sin x$ and $y=\cos x$ for $0\le x\le \pi/2$ and find the
  area enclosed by them and the $x$-axis.


  \problem Write a definite integral whose value is the area inside the
  ellipse $\ds \frac{x^2}{ a^2}+\frac{y^2}{ b^2}=1$.  Evaluate this area.
  Hint: After a suitable change of variable, the definite integral becomes
  the definite integral whose value is the area of a circle.


  \problem Using integration find the area of the triangle with vertices
  $(-1,1)$, $(0,5)$ and $(3,2)$.


  \problem Find the volume that results by rotating the triangle $1\le x\le
  2$, $0\le y\le 3x-3$ around the $x$ axis.




  \problem \label{SphereVolume} Find the volume that results by rotating the
  semicircle $y=\sqrt{R^2-x^2}$ about the $x$-axis.



  \problem A triangle is formed by drawing lines from the two endpoints of a
  line segment of length $b$ to a vertex $V$ which is at a height $h$ above
  the line of the line segment. Its area is then $A=\int_{y=0}^h dA$ where
  $dA$ is the area of the strip cut out by two parallel lines separated by a
  distance of $dz$ and at a height of $z$ above the line containing the line
  segment. Find a formula for $dA$ in terms of $b$, $z$, and $dz$ and
  evaluate the definite integral.


  \problem A pyramid is formed by drawing lines from the four vertices of a
  rectangle of area $A$ to an apex $P$ which is at a height $h$ above the
  plane of the rectangle. Its volume is then $V=\int_{z=0}^h dV$ where $dV$
  is the volume of the slice cut out by two planes parallel to the plane of
  the rectangle and separated by a distance of $dz$ and at a height of $z$
  above the plane of the rectangle. Find a formula for $dV$ in terms of $A$,
  $z$, and $dz$ and evaluate the definite integral.


  \problem A tetrahedron is formed by drawing lines from the three vertices
  of a triangle of area $A$ to an apex $P$ which is at a height $h$ above the
  plane of the triangle. Its volume is then $V=\int_{z=0}^h dV$ where $dV$ is
  the volume of the slice cut out by two planes parallel to the plane of the
  triangle and separated by a distance of $dz$ and at a height of $z$ above
  the plane of the rectangle. Find a formula for $dV$ in terms of $A$, $z$,
  and $dz$ and evaluate the definite integral.


  \problem A cone is formed by drawing lines from the perimeter of a circle
  of area $A$ to an apex $V$ which is at a height $h$ above the plane of the
  circle. Its volume is then $V=\int_{z=0}^h dV$ where $dV$ is the volume of
  the slice cut out by two planes parallel to the plane of the circle and
  separated by a distance of $dz$ and at a height of $z$ above the plane of
  the rectangle. Find a formula for $dV$ in terms of $A$, $z$, and $dz$ and
  evaluate the definite integral.


  \problem (a)~A hemispherical bowl of radius $a$ contains water to a depth
  $h$. Find the volume of the water in the bowl.  (b)~Water runs into a
  hemispherical bowl of radius 5 ft at the rate of 0.2 ft${}^3$/sec. How fast
  is the water level rising when the water is 4 ft deep?


  \problem (Alternate wording for previous problem.)  A hemispherical bowl is
  obtained by rotating the semicircle $x^2+(y-a)^2=a^2$, $y\le a$ about the
  $y$-axis.  It is filled with water to a depth of $h$, i.e. the water level
  is the line $y=h$. (a)~Find the volume of the water in the bowl as a
  function of $h$. (b)~Water runs into a hemispherical bowl of radius 5 ft at
  the rate of 0.2 ft${}^3$/sec. How fast is the water level rising when the
  water is 4 ft deep?  (Hint: Use the method of related rates and the
  Fundamental Theorem.)


  \problem A vase is constructed by rotating the curve $x^3-y^3=1$ for $0\le
  y\le 8$ around the $y$ axis.  It is filled with water to a height $y=h$
  where $h<8$.  (a)~Find the volume of the water in terms of $h$.  (Express
  your answer as a definite integral.  Do not try to evaluate the integral.)
  (b)~If the vase is filling with water at the rate of 2 cubic units per
  second, how fast is the height of the water increasing when this height is
  $2$ units?

\end{multicols}
\noproblemfont

